\chapter{ANSI - Control de Terminal}
\label{ch:ansi}

\section{Introducción}

La librería ANSI proporciona control completo del terminal mediante secuencias de escape ANSI y caracteres Unicode para gráficos.

\begin{lstlisting}
SOURCE "lib/ANSI.BLB"
\end{lstlisting}

\section{Control de Pantalla}

\begin{lstlisting}
ANSI_CLEAR              # Limpiar pantalla
ANSI_CURSOR_HOME        # Cursor a (1,1)
ANSI_CURSOR_GOTO 10 20  # Mover cursor a fila 10, col 20
ANSI_CURSOR_HIDE        # Ocultar cursor
ANSI_CURSOR_SHOW        # Mostrar cursor
ANSI_RESET             # Reset total de formato
\end{lstlisting}

\section{Colores}

\subsection{Constantes de Color}

\textbf{Foreground (texto):}
\begin{itemize}
    \item \texttt{ANSI\_FG\_BLACK}, \texttt{ANSI\_FG\_RED}, \texttt{ANSI\_FG\_GREEN}, \texttt{ANSI\_FG\_YELLOW}
    \item \texttt{ANSI\_FG\_BLUE}, \texttt{ANSI\_FG\_MAGENTA}, \texttt{ANSI\_FG\_CYAN}, \texttt{ANSI\_FG\_WHITE}
    \item Versiones brillantes: \texttt{ANSI\_FG\_BRIGHT\_*}
\end{itemize}

\textbf{Background (fondo):}
\begin{itemize}
    \item \texttt{ANSI\_BG\_*} (mismo esquema)
\end{itemize}

\subsection{Uso}

\begin{lstlisting}
ANSI_SET_COLOR $ANSI_FG_BRIGHT_CYAN $ANSI_BG_BLUE
PUTS "Texto colorido"
ANSI_RESET
\end{lstlisting}

\section{Caracteres Unicode v2.0}

\subsection{Box Drawing - Línea Simple}

\begin{center}
\begin{tabular}{|l|l|c|}
\hline
\textbf{Constante} & \textbf{Descripción} & \textbf{Carácter} \\
\hline
\texttt{ANSI\_BOX\_TL} & Esquina superior izquierda & ┌ \\
\texttt{ANSI\_BOX\_TR} & Esquina superior derecha & ┐ \\
\texttt{ANSI\_BOX\_BL} & Esquina inferior izquierda & └ \\
\texttt{ANSI\_BOX\_BR} & Esquina inferior derecha & ┘ \\
\texttt{ANSI\_BOX\_H} & Línea horizontal & ─ \\
\texttt{ANSI\_BOX\_V} & Línea vertical & │ \\
\hline
\end{tabular}
\end{center}

\subsection{Box Drawing - Línea Doble}

\begin{lstlisting}
ANSI_BOX_D_TL    # ╔
ANSI_BOX_D_TR    # ╗
ANSI_BOX_D_BL    # ╚
ANSI_BOX_D_BR    # ╝
ANSI_BOX_D_H     # ═
ANSI_BOX_D_V     # ║
\end{lstlisting}

\subsection{Box Drawing - Redondeado}

\begin{lstlisting}
ANSI_BOX_R_TL    # ╭
ANSI_BOX_R_TR    # ╮
ANSI_BOX_R_BL    # ╰
ANSI_BOX_R_BR    # ╯
\end{lstlisting}

\subsection{Otros Símbolos}

\begin{lstlisting}
ANSI_ARROW_UP, ANSI_ARROW_DOWN, ANSI_ARROW_LEFT, ANSI_ARROW_RIGHT
ANSI_CHECKMARK   # ✓
ANSI_CROSSMARK   # ✗
ANSI_BULLET      # •
ANSI_STAR        # ★
ANSI_BLOCK_FULL, ANSI_BLOCK_LIGHT, ANSI_BLOCK_MEDIUM, ANSI_BLOCK_DARK
\end{lstlisting}

\section{Ejemplo: Dibujar Caja}

\begin{lstlisting}
SOURCE "lib/ANSI.BLB"

ANSI_CLEAR
ANSI_CURSOR_GOTO 5 10
ANSI_SET_COLOR $ANSI_FG_BRIGHT_CYAN $ANSI_BG_BLACK

# Línea superior
PUTS -NONEWLINE $ANSI_BOX_TL
FOR 0 TO 38 DO
    PUTS -NONEWLINE $ANSI_BOX_H
END
PUTS $ANSI_BOX_TR

# Líneas laterales
FOR 6 TO 14 DO
    SET row $__FOR
    ANSI_CURSOR_GOTO $row 10
    PUTS -NONEWLINE $ANSI_BOX_V
    ANSI_CURSOR_GOTO $row 50
    PUTS $ANSI_BOX_V
END

# Línea inferior
ANSI_CURSOR_GOTO 15 10
PUTS -NONEWLINE $ANSI_BOX_BL
FOR 0 TO 38 DO
    PUTS -NONEWLINE $ANSI_BOX_H
END
PUTS $ANSI_BOX_BR

ANSI_RESET
\end{lstlisting}
