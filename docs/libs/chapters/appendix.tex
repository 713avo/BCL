\appendix

\chapter{Referencia Rápida}

\section{Resumen de Librerías}

\begin{center}
\begin{tabular}{|l|l|l|}
\hline
\textbf{Librería} & \textbf{Archivo} & \textbf{Funciones} \\
\hline
ANSI & ANSI.BLB & Control terminal, colores, Unicode \\
WINDOW & WINDOW.BLB & Ventanas, menús, widgets \\
MATRIX & MATRIX.BLB & Operaciones matriciales \\
CALCULUS & CALCULUS.BLB & Cálculo numérico \\
\hline
\end{tabular}
\end{center}

\section{Tabla de Funciones por Librería}

\subsection{ANSI (21 constantes + funciones)}

\textbf{Control:} CLEAR, CURSOR\_HOME, CURSOR\_GOTO, CURSOR\_HIDE, CURSOR\_SHOW, RESET

\textbf{Colores:} 16 FG + 16 BG, SET\_COLOR, SET\_STYLE

\textbf{Unicode:} 60+ caracteres (box drawing, flechas, símbolos)

\subsection{WINDOW (26 funciones)}

CREATE, DRAW, PRINT, PRINT\_CENTER, SET\_COLOR, SET\_BORDER\_STYLE, SET\_BORDER\_COLOR, CLEAR, HIDE, SHOW, MOVE, MENU, PROGRESS, BUTTON, HLINE, MESSAGE\_BOX, ADD\_LINE, REDRAW\_CONTENT, SCROLL, CLEANUP, INIT

\subsection{MATRIX (23 funciones)}

CREATE, ZEROS, ONES, EYE, RAND, FROM\_LIST, PRINT, ADD, SUB, MUL, SCALAR\_MUL, ELEM\_MUL, TRANSPOSE, SUM, MEAN, MIN, MAX, TRACE, DET\_2X2, COPY, FILL, GET\_ROW, GET\_COL

\subsection{CALCULUS (27 funciones)}

\textbf{Derivadas:} DERIV\_FORWARD, DERIV\_BACKWARD, DERIV\_CENTRAL, DERIV2

\textbf{Integración:} INTEGRATE\_TRAP, INTEGRATE\_SIMPSON, INTEGRATE\_MIDPOINT

\textbf{EDOs:} EULER, RK2, RK4

\textbf{Raíces:} NEWTON, BISECTION, SECANT

\textbf{Polinomios:} POLY\_EVAL, POLY\_ROOTS\_QUAD

\textbf{Series:} LIMIT, SERIES\_SUM

\textbf{Utilidades:} FACTORIAL, COMBINATION, GCD

\section{Caracteres Unicode Box Drawing}

\begin{center}
\begin{tabular}{|l|c|c|c|}
\hline
\textbf{Estilo} & \textbf{Horizontal} & \textbf{Vertical} & \textbf{Esquinas} \\
\hline
Simple & ─ & │ & ┌ ┐ └ ┘ \\
Doble & ═ & ║ & ╔ ╗ ╚ ╝ \\
Redondeado & ─ & │ & ╭ ╮ ╰ ╯ \\
\hline
\end{tabular}
\end{center}

\section{Valores Predeterminados}

\begin{lstlisting}
# CALCULUS
CALC_EPSILON = 1e-6
CALC_MAX_ITER = 1000
CALC_PI = 3.14159265358979323846
CALC_E = 2.71828182845904523536

# WINDOW
WIN_COUNT = 0 (incremental)
\end{lstlisting}

\chapter{Ejemplos Adicionales}

\section{Gráfico de Función con Ventanas}

\begin{lstlisting}
#!/usr/bin/env bcl
SOURCE "lib/WINDOW.BLB"
SOURCE "lib/CALCULUS.BLB"

WIN_INIT

# Ventana para gráfico
WIN_CREATE 0 2 2 76 22 "Gráfico de f(x) = sin(x)"
WIN_SET_BORDER_STYLE 0 1
WIN_DRAW 0

# Generar puntos
PROC GENERATE_PLOT DO
    GLOBAL RK4_X RK4_Y
    # Usar RK4 para resolver dy/dx = cos(x), y(0)=0
    # Solución: y = sin(x)
    CALC_RK4 "cos($x)" 0 0 0.1 63

    FOR 0 TO 62 DO
        SET i $__FOR
        SET x $RK4_X($i)
        SET y $RK4_Y($i)

        # Escalar a coordenadas de ventana
        SET row [EXPR 11 - int($y * 10)]
        SET col [EXPR 2 + $i]

        IF [EXPR $row >= 1 && $row <= 20] THEN
            ANSI_CURSOR_GOTO [EXPR $row + 2] [EXPR $col + 2]
            PUTS -NONEWLINE "*"
        END
    END
END

GENERATE_PLOT

ANSI_CURSOR_GOTO 24 1
PUTS -NONEWLINE "Presione ENTER..."
GETS input

WIN_CLEANUP
\end{lstlisting}

\section{Calculadora Matricial Interactiva}

\begin{lstlisting}
#!/usr/bin/env bcl
SOURCE "lib/WINDOW.BLB"
SOURCE "lib/MATRIX.BLB"

PROC CALC_MATRIX_INTERACTIVE DO
    GLOBAL M1 M2 result

    WIN_INIT

    # Ventana principal
    WIN_CREATE 0 2 5 70 20 "Calculadora Matricial"
    WIN_SET_BORDER_STYLE 0 1
    WIN_DRAW 0

    # Crear matrices de ejemplo
    MAT_FROM_LIST M1 2 2 "1 2 3 4"
    MAT_FROM_LIST M2 2 2 "5 6 7 8"

    WIN_PRINT 0 1 "Matriz A:"
    WIN_PRINT 0 2 "  [1 2]"
    WIN_PRINT 0 3 "  [3 4]"

    WIN_PRINT 0 5 "Matriz B:"
    WIN_PRINT 0 6 "  [5 6]"
    WIN_PRINT 0 7 "  [7 8]"

    # Menú de operaciones
    WIN_PRINT 0 9 "Operación:"
    SET ops "A+B|A-B|A*B|Det(A)|Salir"
    WIN_MENU 0 $ops 1

    # Realizar operación (simplificado - selección fija)
    MAT_ADD M1 M2 result
    WIN_PRINT 0 14 "Resultado (A+B):"
    WIN_PRINT 0 15 "  [6 8]"
    WIN_PRINT 0 16 "  [10 12]"

    WIN_HLINE 0 17
    WIN_BUTTON 0 18 25 " Calcular "

    ANSI_CURSOR_GOTO 24 1
    PUTS -NONEWLINE "Presione ENTER..."
    GETS input

    WIN_CLEANUP
END

CALC_MATRIX_INTERACTIVE
\end{lstlisting}

\chapter{Solución de Problemas}

\section{Errores Comunes}

\subsection{Arrays no persisten en procedimientos}

\textbf{Error:} "no such variable"

\textbf{Solución:} Declarar arrays como GLOBAL

\begin{lstlisting}
PROC MAL DO
    MAT_ZEROS M 3 3    # ERROR
END

PROC BIEN DO
    GLOBAL M           # CORRECTO
    MAT_ZEROS M 3 3
END
\end{lstlisting}

\subsection{Terminal no muestra Unicode}

\textbf{Síntoma:} Caracteres extraños en lugar de box drawing

\textbf{Solución:}
\begin{enumerate}
    \item Verificar terminal soporta UTF-8
    \item Configurar locale: \texttt{export LANG=es\_ES.UTF-8}
    \item Usar terminal moderna (xterm, gnome-terminal, etc.)
\end{enumerate}

\subsection{Métodos numéricos no convergen}

\textbf{Solución:}
\begin{itemize}
    \item Verificar condiciones iniciales
    \item Ajustar tolerancia (\texttt{CALC\_EPSILON})
    \item Aumentar iteraciones máximas
    \item Verificar que la función tiene raíz/solución
\end{itemize}

\section{Optimización}

\subsection{Performance de Matrices}

Para matrices grandes, considerar:
\begin{itemize}
    \item Usar operaciones elemento a elemento cuando sea posible
    \item Evitar copias innecesarias
    \item BCL no está optimizado para matrices masivas ($>100 \times 100$)
\end{itemize}

\subsection{Integración Numérica}

\begin{itemize}
    \item Simpson es más preciso que trapecio
    \item Para misma precisión, Simpson requiere menos puntos
    \item Elegir \texttt{n} según precisión deseada
\end{itemize}

\section{Recursos Adicionales}

\begin{itemize}
    \item \textbf{Documentación principal:} \texttt{docs/man\_llm.md}
    \item \textbf{Ejemplos:} Carpeta \texttt{examples/}
    \item \textbf{Tests:} Carpeta \texttt{tests/}
    \item \textbf{Código fuente:} \texttt{lib/*.BLB}
\end{itemize}

\vspace{2cm}

\begin{center}
\textbf{Fin del Manual de Librerías BCL v2.0}

\vspace{1cm}

Para más información, visite:\\
\url{https://github.com/yourusername/bcl}
\end{center}
