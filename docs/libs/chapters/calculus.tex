\chapter{CALCULUS - Cálculo Numérico}
\label{ch:calculus}

\section{Introducción}

La librería CALCULUS proporciona un conjunto completo de funciones para cálculo numérico, incluyendo:

\begin{itemize}
    \item Derivadas numéricas (diferencias finitas)
    \item Integración numérica (trapecio, Simpson, punto medio)
    \item Resolución de ecuaciones diferenciales ordinarias (Euler, Runge-Kutta)
    \item Búsqueda de raíces (Newton-Raphson, bisección, secante)
    \item Operaciones con polinomios
    \item Cálculo de límites y series
\end{itemize}

\begin{notebox}
Esta librería implementa métodos numéricos estándar con precisión configurable. Para aplicaciones que requieren alta precisión, ajuste los parámetros \texttt{CALC\_EPSILON} y \texttt{CALC\_MAX\_ITER}.
\end{notebox}

\section{Carga y configuración}

\begin{lstlisting}
SOURCE "lib/CALCULUS.BLB"

# Configuración opcional
SET CALC_EPSILON 1e-8        # Tolerancia (default: 1e-6)
SET CALC_MAX_ITER 5000       # Max iteraciones (default: 1000)

# Constantes disponibles
PUTS $CALC_PI                # π ≈ 3.14159...
PUTS $CALC_E                 # e ≈ 2.71828...
\end{lstlisting}

\section{Derivadas Numéricas}

\subsection{CALC\_DERIV\_FORWARD}

Calcula la derivada usando diferencias finitas hacia adelante:
\[
f'(x) \approx \frac{f(x+h) - f(x)}{h}
\]

\textbf{Sintaxis:}
\begin{lstlisting}
CALC_DERIV_FORWARD expr x h result
\end{lstlisting}

\textbf{Parámetros:}
\begin{itemize}
    \item \texttt{expr}: Expresión de la función (debe usar variable \texttt{\$x})
    \item \texttt{x}: Punto donde calcular la derivada
    \item \texttt{h}: Paso (típicamente 0.001 a 0.00001)
    \item \texttt{result}: Variable donde almacenar el resultado
\end{itemize}

\textbf{Ejemplo:}
\begin{lstlisting}
# Derivada de f(x) = x^2 en x=3
# f'(x) = 2x, entonces f'(3) = 6
CALC_DERIV_FORWARD "$x * $x" 3.0 0.001 deriv
PUTS "f'(3) ≈ $deriv"
# Output: f'(3) ≈ 6.001
\end{lstlisting}

\subsection{CALC\_DERIV\_CENTRAL}

Derivada con diferencias centrales (más precisa):
\[
f'(x) \approx \frac{f(x+h) - f(x-h)}{2h}
\]

\textbf{Ejemplo:}
\begin{lstlisting}
# Derivada de sin(x) en x=π/2
# d/dx[sin(x)] = cos(x), cos(π/2) = 0
SET x_val [EXPR $CALC_PI / 2.0]
CALC_DERIV_CENTRAL "sin($x)" $x_val 0.0001 deriv
PUTS "cos(π/2) ≈ $deriv"
# Output: cos(π/2) ≈ 0.000000
\end{lstlisting}

\subsection{CALC\_DERIV2}

Segunda derivada usando diferencias finitas:
\[
f''(x) \approx \frac{f(x+h) - 2f(x) + f(x-h)}{h^2}
\]

\textbf{Ejemplo:}
\begin{lstlisting}
# Segunda derivada de x^3
# f(x) = x^3, f''(x) = 6x, f''(2) = 12
CALC_DERIV2 "$x * $x * $x" 2.0 0.001 deriv2
PUTS "f''(2) ≈ $deriv2"
# Output: f''(2) ≈ 12.000
\end{lstlisting}

\section{Integración Numérica}

\subsection{CALC\_INTEGRATE\_TRAP}

Integración por regla del trapecio:
\[
\int_a^b f(x)dx \approx \frac{h}{2}\left[f(a) + 2\sum_{i=1}^{n-1}f(x_i) + f(b)\right]
\]

\textbf{Sintaxis:}
\begin{lstlisting}
CALC_INTEGRATE_TRAP expr a b n result
\end{lstlisting}

\textbf{Parámetros:}
\begin{itemize}
    \item \texttt{expr}: Expresión del integrando
    \item \texttt{a}, \texttt{b}: Límites de integración
    \item \texttt{n}: Número de subintervalos (mayor = más preciso)
    \item \texttt{result}: Variable resultado
\end{itemize}

\textbf{Ejemplo:}
\begin{lstlisting}
# Integral de x^2 de 0 a 1
# ∫₀¹ x² dx = [x³/3]₀¹ = 1/3 ≈ 0.333333
CALC_INTEGRATE_TRAP "$x * $x" 0 1 100 area
PUTS "Área = $area"
# Output: Área = 0.333333
\end{lstlisting}

\subsection{CALC\_INTEGRATE\_SIMPSON}

Regla de Simpson (más precisa que trapecio):
\[
\int_a^b f(x)dx \approx \frac{h}{3}\left[f(a) + 4\sum f_{odd} + 2\sum f_{even} + f(b)\right]
\]

\textbf{Nota:} \texttt{n} debe ser par. Si es impar, se incrementa automáticamente.

\textbf{Ejemplo:}
\begin{lstlisting}
# Integral de sin(x) de 0 a π
# ∫₀^π sin(x) dx = [-cos(x)]₀^π = 2
CALC_INTEGRATE_SIMPSON "sin($x)" 0 $CALC_PI 100 result
PUTS "Resultado = $result"
# Output: Resultado ≈ 2.000000
\end{lstlisting}

\subsection{Comparación de métodos}

\begin{center}
\begin{tabular}{|l|l|l|}
\hline
\textbf{Método} & \textbf{Precisión} & \textbf{Uso recomendado} \\
\hline
Trapecio & $O(h^2)$ & Integrales simples, n grande \\
Simpson & $O(h^4)$ & Mayor precisión, funciones suaves \\
Punto medio & $O(h^2)$ & Similar a trapecio \\
\hline
\end{tabular}
\end{center}

\section{Ecuaciones Diferenciales}

\subsection{CALC\_EULER}

Método de Euler para resolver EDOs de la forma $dy/dx = f(x,y)$.

\textbf{Sintaxis:}
\begin{lstlisting}
PROC MY_SOLVER DO
    GLOBAL EULER_X EULER_Y    # Arrays de resultados
    CALC_EULER expr x0 y0 h n
END
\end{lstlisting}

\textbf{Parámetros:}
\begin{itemize}
    \item \texttt{expr}: Expresión $f(x,y)$ (usa variables \texttt{\$x} e \texttt{\$y})
    \item \texttt{x0}, \texttt{y0}: Condición inicial $y(x_0) = y_0$
    \item \texttt{h}: Paso de integración
    \item \texttt{n}: Número de pasos
\end{itemize}

\textbf{Resultado:} Arrays globales \texttt{EULER\_X(i)} y \texttt{EULER\_Y(i)} con \texttt{i=0..n}

\textbf{Ejemplo:}
\begin{lstlisting}
# Resolver dy/dx = y, y(0) = 1
# Solución exacta: y = e^x
PROC SOLVE_EXPONENTIAL DO
    GLOBAL EULER_X EULER_Y
    CALC_EULER "$y" 0 1 0.1 10

    # Imprimir resultados
    FOR 0 TO 10 DO
        SET i $__FOR
        PUTS "x=$EULER_X($i), y=$EULER_Y($i)"
    END
END
SOLVE_EXPONENTIAL
\end{lstlisting}

\subsection{CALC\_RK4}

Método de Runge-Kutta de 4º orden (alta precisión).

\begin{notebox}
RK4 es significativamente más preciso que Euler para el mismo paso \texttt{h}. Recomendado para problemas que requieren alta exactitud.
\end{notebox}

\textbf{Sintaxis:}
\begin{lstlisting}
PROC MY_SOLVER DO
    GLOBAL RK4_X RK4_Y
    CALC_RK4 expr x0 y0 h n
END
\end{lstlisting}

\textbf{Ejemplo:}
\begin{lstlisting}
# Resolver dy/dx = -2xy, y(0) = 1
# (Decaimiento Gaussiano)
PROC SOLVE_GAUSSIAN DO
    GLOBAL RK4_X RK4_Y
    CALC_RK4 "-2 * $x * $y" 0 1 0.1 20

    # Valor final
    PUTS "y(2.0) ≈ $RK4_Y(20)"
END
SOLVE_GAUSSIAN
\end{lstlisting}

\section{Búsqueda de Raíces}

\subsection{CALC\_NEWTON}

Método de Newton-Raphson para encontrar raíces.

\textbf{Fórmula:}
\[
x_{n+1} = x_n - \frac{f(x_n)}{f'(x_n)}
\]

\textbf{Sintaxis:}
\begin{lstlisting}
CALC_NEWTON f_expr df_expr x0 tol max_iter result
\end{lstlisting}

\textbf{Parámetros:}
\begin{itemize}
    \item \texttt{f\_expr}: Expresión de $f(x)$
    \item \texttt{df\_expr}: Expresión de $f'(x)$
    \item \texttt{x0}: Estimación inicial
    \item \texttt{tol}: Tolerancia (ej. 1e-6)
    \item \texttt{max\_iter}: Máximo de iteraciones
    \item \texttt{result}: Variable resultado
\end{itemize}

\textbf{Ejemplo:}
\begin{lstlisting}
# Encontrar √2 resolviendo x^2 - 2 = 0
CALC_NEWTON "$x*$x - 2" "2*$x" 1.0 1e-8 100 root
PUTS "√2 ≈ $root"
# Output: √2 ≈ 1.41421356
\end{lstlisting}

\subsection{CALC\_BISECTION}

Método de la bisección (más robusto, convergencia garantizada).

\textbf{Requisito:} $f(a) \cdot f(b) < 0$ (signos opuestos).

\textbf{Sintaxis:}
\begin{lstlisting}
CALC_BISECTION expr a b tol max_iter result
\end{lstlisting}

\textbf{Ejemplo:}
\begin{lstlisting}
# Encontrar raíz de cos(x) en [0, π/2]
CALC_BISECTION "cos($x)" 0 [EXPR $CALC_PI/2] 1e-6 100 root
PUTS "Raíz ≈ $root"
# Output: Raíz ≈ 1.570796 (π/2)
\end{lstlisting}

\subsection{CALC\_SECANT}

Método de la secante (no requiere derivada).

\textbf{Fórmula:}
\[
x_{n+1} = x_n - f(x_n)\frac{x_n - x_{n-1}}{f(x_n) - f(x_{n-1})}
\]

\textbf{Sintaxis:}
\begin{lstlisting}
CALC_SECANT expr x0 x1 tol max_iter result
\end{lstlisting}

\textbf{Ejemplo:}
\begin{lstlisting}
# Encontrar raíz de x^3 - x - 1 = 0
CALC_SECANT "$x*$x*$x - $x - 1" 1.0 2.0 1e-6 100 root
PUTS "Raíz ≈ $root"
# Output: Raíz ≈ 1.324718
\end{lstlisting}

\subsection{Comparación de métodos}

\begin{center}
\begin{tabular}{|l|l|l|l|}
\hline
\textbf{Método} & \textbf{Convergencia} & \textbf{Derivada} & \textbf{Robustez} \\
\hline
Newton & Cuadrática & Requiere & Media \\
Bisección & Lineal & No requiere & Alta \\
Secante & Superlineal & No requiere & Media \\
\hline
\end{tabular}
\end{center}

\section{Polinomios}

\subsection{CALC\_POLY\_EVAL}

Evalúa polinomio usando el método de Horner.

\textbf{Sintaxis:}
\begin{lstlisting}
PROC EVAL_POLYNOMIAL DO
    GLOBAL coeffs
    SET coeffs(0) 1     # a₀
    SET coeffs(1) 2     # a₁
    SET coeffs(2) 3     # a₂
    # P(x) = 1 + 2x + 3x²

    CALC_POLY_EVAL 2 5.0 result
    PUTS "P(5) = $result"
    # P(5) = 1 + 2(5) + 3(25) = 86
END
\end{lstlisting}

\subsection{CALC\_POLY\_ROOTS\_QUAD}

Encuentra raíces de ecuación cuadrática $ax^2 + bx + c = 0$.

\textbf{Sintaxis:}
\begin{lstlisting}
PROC FIND_ROOTS DO
    GLOBAL POLY_ROOT1 POLY_ROOT2 POLY_DISCRIMINANT
    CALC_POLY_ROOTS_QUAD a b c
END
\end{lstlisting}

\textbf{Ejemplo:}
\begin{lstlisting}
# Resolver x^2 - 5x + 6 = 0
# Raíces: x = 2, x = 3
PROC SOLVE_QUADRATIC DO
    GLOBAL POLY_ROOT1 POLY_ROOT2 POLY_DISCRIMINANT
    CALC_POLY_ROOTS_QUAD 1 -5 6
    PUTS "x₁ = $POLY_ROOT1"
    PUTS "x₂ = $POLY_ROOT2"
END
SOLVE_QUADRATIC
\end{lstlisting}

\section{Series y Límites}

\subsection{CALC\_SERIES\_SUM}

Calcula suma de serie $\sum_{n=start}^{end} f(n)$.

\textbf{Ejemplo:}
\begin{lstlisting}
# Aproximar π²/6 con serie: Σ(1/n²) n=1 to ∞
CALC_SERIES_SUM "1.0 / ($n * $n)" 1 10000 sum
SET pi_sq_6 [EXPR $sum]
SET pi_approx [EXPR sqrt($pi_sq_6 * 6.0)]
PUTS "π ≈ $pi_approx"
\end{lstlisting}

\subsection{CALC\_LIMIT}

Estimación numérica de límites.

\textbf{Ejemplo:}
\begin{lstlisting}
# lim(x→0) sin(x)/x = 1
CALC_LIMIT "sin($x) / $x" 0 limit
PUTS "límite = $limit"
# Output: límite ≈ 1.000000
\end{lstlisting}

\section{Funciones Utilidad}

\subsection{CALC\_FACTORIAL}

\begin{lstlisting}
CALC_FACTORIAL 5 result
PUTS "5! = $result"
# Output: 5! = 120
\end{lstlisting}

\subsection{CALC\_COMBINATION}

Calcula combinaciones $C(n,k) = \frac{n!}{k!(n-k)!}$.

\begin{lstlisting}
CALC_COMBINATION 10 3 result
PUTS "C(10,3) = $result"
# Output: C(10,3) = 120
\end{lstlisting}

\subsection{CALC\_GCD}

Máximo común divisor (algoritmo de Euclides).

\begin{lstlisting}
CALC_GCD 48 18 result
PUTS "GCD(48,18) = $result"
# Output: GCD(48,18) = 6
\end{lstlisting}

\section{Ejemplo Completo: Análisis de Función}

\begin{lstlisting}
#!/usr/bin/env bcl
SOURCE "lib/CALCULUS.BLB"

# Analizar f(x) = x³ - 3x² + 2x
PUTS "Análisis de f(x) = x³ - 3x² + 2x"
PUTS "================================="

# 1. Encontrar derivada en x=1
CALC_DERIV_CENTRAL "$x*$x*$x - 3*$x*$x + 2*$x" 1.0 0.0001 deriv
PUTS "f'(1) = $deriv"

# 2. Calcular segunda derivada en x=1
CALC_DERIV2 "$x*$x*$x - 3*$x*$x + 2*$x" 1.0 0.0001 deriv2
PUTS "f''(1) = $deriv2"

# 3. Encontrar raíz cerca de x=2
CALC_NEWTON "$x*$x*$x - 3*$x*$x + 2*$x" \
            "3*$x*$x - 6*$x + 2" 2.0 1e-8 100 root
PUTS "Raíz encontrada: x = $root"

# 4. Calcular área bajo la curva [0,1]
CALC_INTEGRATE_SIMPSON "$x*$x*$x - 3*$x*$x + 2*$x" \
                        0 1 100 area
PUTS "Área [0,1] = $area"

PUTS "================================="
\end{lstlisting}

\section{Notas de Precisión}

\begin{warningbox}
Los métodos numéricos tienen limitaciones inherentes:
\begin{itemize}
    \item \textbf{Error de redondeo}: Acumulado en operaciones flotantes
    \item \textbf{Error de truncamiento}: Por aproximación de derivadas/integrales
    \item \textbf{Convergencia}: No garantizada para todos los problemas
\end{itemize}

Recomendaciones:
\begin{itemize}
    \item Usar pasos pequeños (\texttt{h}) para derivadas/integrales
    \item Verificar convergencia en métodos iterativos
    \item Comparar resultados con diferentes parámetros
    \item Para alta precisión, considerar librerías especializadas
\end{itemize}
\end{warningbox}
