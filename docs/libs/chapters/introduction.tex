\chapter{Introducción}
\label{ch:introduction}

\section{Acerca de este documento}

Este manual describe las librerías estándar incluidas en BCL (Basic Command Language) versión 2.0 y posteriores. Las librerías estándar proporcionan funcionalidad avanzada para:

\begin{itemize}
    \item Control de terminal y gráficos ANSI
    \item Gestión de ventanas en terminal
    \item Operaciones matriciales estilo MATLAB
    \item Cálculo numérico y análisis matemático
\end{itemize}

\section{Requisitos}

\subsection{Versión de BCL}

Todas las librerías descritas en este documento requieren BCL versión 2.0.0 o superior. Esta versión incluye mejoras fundamentales:

\begin{itemize}
    \item \textbf{Sistema de arrays globales}: Los arrays ahora funcionan correctamente con la declaración \texttt{GLOBAL} en procedimientos
    \item \textbf{Soporte Unicode completo}: Secuencias de escape \texttt{\textbackslash uXXXX} y \texttt{\textbackslash UXXXXXXXX} para UTF-8
    \item \textbf{Librerías mejoradas}: ANSI y WINDOW completamente reescritas, MATRIX y CALCULUS nuevas
\end{itemize}

\subsection{Terminal compatible}

Para aprovechar las funcionalidades de ANSI y WINDOW, se requiere un emulador de terminal que soporte:

\begin{itemize}
    \item Secuencias de escape ANSI
    \item Codificación UTF-8
    \item Caracteres Unicode box-drawing
    \item Colores de 16 bits (8 básicos + 8 brillantes)
\end{itemize}

Terminales recomendadas: xterm, gnome-terminal, konsole, iTerm2, Windows Terminal.

\section{Instalación}

Las librerías estándar se incluyen en la distribución de BCL en el directorio \texttt{lib/}:

\begin{verbatim}
BCL/
 +-- lib/
     |-- ANSI.BLB       # Terminal control y gráficos
     |-- WINDOW.BLB     # Gestión de ventanas
     |-- MATRIX.BLB     # Operaciones matriciales
     +-- CALCULUS.BLB   # Cálculo numérico
\end{verbatim}

No se requiere instalación adicional. Las librerías están listas para usar mediante el comando \texttt{SOURCE}.

\section{Uso básico}

\subsection{Cargar una librería}

Para utilizar una librería en un script BCL, use el comando \texttt{SOURCE}:

\begin{lstlisting}
#!/usr/bin/env bcl

SOURCE "lib/ANSI.BLB"
SOURCE "lib/WINDOW.BLB"
SOURCE "lib/MATRIX.BLB"
SOURCE "lib/CALCULUS.BLB"
\end{lstlisting}

\begin{notebox}
Las rutas de las librerías son relativas al directorio de trabajo actual. Si ejecuta scripts desde otros directorios, ajuste la ruta según sea necesario.
\end{notebox}

\subsection{Arrays globales en procedimientos}

\begin{warningbox}
\textbf{Importante}: Debido a limitaciones actuales de BCL, los arrays utilizados por las librerías MATRIX y algunas funciones de otras librerías deben declararse como \texttt{GLOBAL} cuando se usen dentro de procedimientos.
\end{warningbox}

Ejemplo correcto:

\begin{lstlisting}
PROC MI_FUNCION DO
    GLOBAL M A B resultado    # Declarar arrays como globales
    MAT_ZEROS M 3 3
    MAT_ADD A B resultado
END
\end{lstlisting}

Ejemplo incorrecto (no funcionará):

\begin{lstlisting}
PROC MI_FUNCION DO
    # Falta GLOBAL - los arrays no persistirán
    MAT_ZEROS M 3 3          # ERROR
    MAT_ADD A B resultado    # ERROR
END
\end{lstlisting}

\section{Convenciones}

\subsection{Nomenclatura}

Todas las funciones de librería siguen estas convenciones:

\begin{itemize}
    \item \textbf{Prefijo por librería}: Las funciones se agrupan por prefijo
        \begin{itemize}
            \item \texttt{ANSI\_}: Librería ANSI
            \item \texttt{WIN\_}: Librería WINDOW
            \item \texttt{MAT\_}: Librería MATRIX
            \item \texttt{CALC\_}: Librería CALCULUS
        \end{itemize}
    \item \textbf{Mayúsculas}: Todas las funciones en mayúsculas (aunque BCL es case-insensitive)
    \item \textbf{Guión bajo}: Separador de palabras en nombres compuestos
\end{itemize}

\subsection{Parámetros}

\begin{itemize}
    \item \textbf{Entrada}: Parámetros que la función lee
    \item \textbf{Salida}: Parámetros donde la función escribe resultados (usualmente el último parámetro)
    \item \textbf{Opcionales}: Marcados con \texttt{[opcional]} en la documentación
\end{itemize}

\subsection{Códigos de ejemplo}

Los ejemplos de código en este manual están formateados para claridad:

\begin{examplebox}
Ejemplo de código con resultado:
\begin{lstlisting}
SET a 5
SET b 3
SET suma [EXPR $a + $b]
PUTS "Suma: $suma"
\end{lstlisting}

Salida:
\begin{verbatim}
Suma: 8
\end{verbatim}
\end{examplebox}

\section{Estructura del documento}

Este manual está organizado en los siguientes capítulos:

\begin{description}
    \item[Capítulo 2: ANSI] Control de terminal, colores, cursores y caracteres Unicode
    \item[Capítulo 3: WINDOW] Sistema de ventanas en terminal con menús y controles
    \item[Capítulo 4: MATRIX] Operaciones matriciales estilo MATLAB
    \item[Capítulo 5: CALCULUS] Cálculo numérico: derivadas, integrales, métodos numéricos
    \item[Apéndices] Referencia rápida, tablas de funciones y ejemplos completos
\end{description}

\section{Soporte y contribuciones}

Para reportar problemas, sugerir mejoras o contribuir al desarrollo de las librerías:

\begin{itemize}
    \item \textbf{GitHub}: \url{https://github.com/yourusername/bcl}
    \item \textbf{Documentación}: \texttt{docs/man\_llm.md}
    \item \textbf{Ejemplos}: Carpeta \texttt{examples/}
\end{itemize}

\section{Licencia}

Las librerías estándar de BCL se distribuyen bajo la misma licencia que el intérprete BCL. Consulte el archivo \texttt{LICENSE.txt} para más detalles.
