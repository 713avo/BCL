\chapter{MATRIX - Operaciones Matriciales}
\label{ch:matrix}

\section{Introducción}

Librería de operaciones matriciales estilo MATLAB para BCL.

\begin{warningbox}
\textbf{Importante}: Las matrices deben declararse como \texttt{GLOBAL} en procedimientos.

\begin{lstlisting}
PROC MI_FUNCION DO
    GLOBAL M A B resultado    # Obligatorio
    MAT_ZEROS M 3 3
END
\end{lstlisting}
\end{warningbox}

\begin{lstlisting}
SOURCE "lib/MATRIX.BLB"
\end{lstlisting}

\section{Creación de Matrices}

\begin{lstlisting}
MAT_ZEROS name rows cols              # Matriz de ceros
MAT_ONES name rows cols               # Matriz de unos
MAT_EYE name size                     # Matriz identidad
MAT_RAND name rows cols max           # Matriz aleatoria [0,max]
MAT_FROM_LIST name rows cols data     # Desde lista plana
\end{lstlisting}

\textbf{Ejemplo:}
\begin{lstlisting}
GLOBAL M I A

MAT_ZEROS M 3 3                       # M = 3x3 de ceros
MAT_EYE I 3                           # I = identidad 3x3
MAT_FROM_LIST A 2 2 "1 2 3 4"         # A = [[1,2],[3,4]]
\end{lstlisting}

\section{Operaciones}

\begin{lstlisting}
MAT_ADD A B C                         # C = A + B
MAT_SUB A B C                         # C = A - B
MAT_MUL A B C                         # C = A * B (multiplicación matricial)
MAT_SCALAR_MUL A k B                  # B = k * A
MAT_ELEM_MUL A B C                    # C = A .* B (elemento a elemento)
MAT_TRANSPOSE A A_T                   # A_T = A'
\end{lstlisting}

\section{Análisis}

\begin{lstlisting}
MAT_SUM A total                       # Suma de todos los elementos
MAT_MEAN A promedio                   # Promedio
MAT_MIN A minimo                      # Valor mínimo
MAT_MAX A maximo                      # Valor máximo
MAT_TRACE A traza                     # Traza (suma diagonal)
MAT_DET_2X2 A det                     # Determinante (solo 2x2)
\end{lstlisting}

\section{Utilidades}

\begin{lstlisting}
MAT_PRINT name                        # Imprimir matriz formateada
MAT_COPY src dst                      # Copiar matriz
MAT_FILL name value                   # Llenar con valor
MAT_GET_ROW name row result           # Extraer fila
MAT_GET_COL name col result           # Extraer columna
\end{lstlisting}

\section{Ejemplo Completo}

\begin{lstlisting}
#!/usr/bin/env bcl
SOURCE "lib/MATRIX.BLB"

PROC MATRIZ_DEMO DO
    GLOBAL A B C I resultado

    # Crear matrices
    MAT_FROM_LIST A 2 2 "1 2 3 4"
    MAT_FROM_LIST B 2 2 "5 6 7 8"
    MAT_EYE I 2

    # Operaciones
    PUTS "A:"
    MAT_PRINT A
    PUTS "\nB:"
    MAT_PRINT B

    MAT_ADD A B C
    PUTS "\nA + B:"
    MAT_PRINT C

    MAT_MUL A B C
    PUTS "\nA * B:"
    MAT_PRINT C

    # Análisis
    MAT_TRACE A tr
    MAT_DET_2X2 A det
    PUTS "\nTr(A) = $tr"
    PUTS "Det(A) = $det"
END

MATRIZ_DEMO
\end{lstlisting}

\section{Almacenamiento}

Las matrices se almacenan en el array global \texttt{\_mat}:

\begin{itemize}
    \item Metadata: \texttt{\_mat(name.rows)}, \texttt{\_mat(name.cols)}
    \item Elementos: \texttt{name(i,j)} donde i=fila, j=columna
    \item Índices base-0
\end{itemize}
