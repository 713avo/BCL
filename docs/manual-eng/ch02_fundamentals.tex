% ============================================================================
% CHAPTER 2: PROGRAMMING FUNDAMENTALS
% ============================================================================

\chapter{Programming Fundamentals}
\label{ch:fundamentals}

This chapter introduces basic programming concepts for absolute beginners using BCL.

\section{What is a Program?}

A program is a sequence of instructions that tells the computer what to do. In BCL, programs consist of commands, each on its own line or separated by semicolons.

\begin{examplebox}[title=Simple Program Structure]
\begin{lstlisting}[language=BCL]
# A program with three commands
PUTS "Starting program..."
SET name "Alice"
PUTS "Hello, $name!"
\end{lstlisting}

\textbf{Output:}
\begin{verbatim}
Starting program...
Hello, Alice!
\end{verbatim}
\end{examplebox}

\section{Variables: Storing Information}

Variables are named containers that store values. Think of them as labeled boxes where you can put information and retrieve it later.

\subsection{Creating Variables}

Use the \cmd{SET} command to create and assign values to variables:

\begin{lstlisting}[language=BCL]
SET age 25
SET name "Bob"
SET price 19.99
\end{lstlisting}

\subsection{Using Variables}

To use a variable's value, prefix its name with a dollar sign (\$):

\begin{examplebox}[title=Using Variables]
\begin{lstlisting}[language=BCL]
SET username "Alice"
SET score 100

PUTS "User: $username"
PUTS "Score: $score"
\end{lstlisting}

\textbf{Output:}
\begin{verbatim}
User: Alice
Score: 100
\end{verbatim}
\end{examplebox}

\section{Data Types: Everything is a String}

Unlike many programming languages, BCL doesn't have separate types for numbers, text, etc. Everything is stored as a string (text). BCL automatically interprets strings as numbers when needed for calculations.

\begin{examplebox}[title=Strings as Numbers]
\begin{lstlisting}[language=BCL]
SET a "10"
SET b "20"
SET sum [EXPR $a + $b]
PUTS "Sum: $sum"
\end{lstlisting}

\textbf{Output:}
\begin{verbatim}
Sum: 30
\end{verbatim}

\textbf{Explanation:} Even though \var{a} and \var{b} are text, \cmd{EXPR} interprets them as numbers for arithmetic.
\end{examplebox}

\section{Comments: Documenting Your Code}

Comments are notes for humans that the computer ignores. They help explain what your code does.

\begin{lstlisting}[language=BCL]
# This is a single-line comment

SET x 10  # Comments can also appear at line end
SET y 20  ;# Using semicolon before hash is also valid
\end{lstlisting}

\begin{tipbox}
Write comments to explain \textit{why} your code does something, not just \textit{what} it does. Future you will thank present you!
\end{tipbox}

\section{Basic Input and Output}

\subsection{Output: PUTS and PUTSN}

The \cmd{PUTS} command displays text on the screen:

\begin{lstlisting}[language=BCL]
PUTS "This appears on the screen"
PUTS "Each PUTS starts a new line"
\end{lstlisting}

Use \cmd{PUTSN} to print without adding a newline:

\begin{examplebox}[title=PUTS vs PUTSN]
\begin{lstlisting}[language=BCL]
PUTS "Line 1"
PUTS "Line 2"
PUTSN "Same "
PUTSN "line"
PUTS ""  # Empty line
\end{lstlisting}

\textbf{Output:}
\begin{verbatim}
Line 1
Line 2
Same line
\end{verbatim}
\end{examplebox}

\subsection{Input: GETS}

The \cmd{GETS} command reads user input from the keyboard:

\begin{examplebox}[title=Reading User Input]
\begin{lstlisting}[language=BCL]
PUTS "What is your name?"
SET name [GETS stdin]
PUTS "Hello, $name!"
\end{lstlisting}

\textbf{Interaction:}
\begin{verbatim}
What is your name?
> Charlie
Hello, Charlie!
\end{verbatim}
\end{examplebox}

\section{Your First Interactive Program}

Let's combine what we've learned:

\begin{examplebox}[title=Interactive Greeting Program]
\begin{lstlisting}[language=BCL]
# Interactive greeting program
PUTS "=== Welcome to BCL ==="
PUTS ""

# Get user's name
PUTS "Please enter your name:"
SET username [GETS stdin]

# Get user's age
PUTS "Please enter your age:"
SET age [GETS stdin]

# Display personalized greeting
PUTS ""
PUTS "Hello, $username!"
PUTS "You are $age years old."
PUTS "Welcome to the world of BCL programming!"
\end{lstlisting}
\end{examplebox}
