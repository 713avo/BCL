% ============================================================================
% CHAPTER 3: VARIABLES AND DATA
% ============================================================================

\chapter{Variables and Data}
\label{ch:variables}

This chapter explores BCL's variable system in depth, including creation, modification, scope, and advanced techniques.

\section{Creating and Modifying Variables}

\subsection{SET: The Foundation}

The \cmd{SET} command is used for both creating and modifying variables.

\textbf{Syntax:}
\begin{lstlisting}[language=BCL]
SET variablename value        # Assign value
SET variablename              # Return current value
\end{lstlisting}

\begin{examplebox}[title=SET Examples]
\begin{lstlisting}[language=BCL]
# Creating variables
SET counter 0
SET message "Hello"
SET pi 3.14159

# Modifying variables
SET counter 10
SET message "Goodbye"

# Reading values
PUTS [SET counter]  # Prints: 10
\end{lstlisting}
\end{examplebox}

\subsection{UNSET: Removing Variables}

The \cmd{UNSET} command deletes a variable from memory.

\begin{examplebox}[title=UNSET Example]
\begin{lstlisting}[language=BCL]
SET temp "temporary data"
PUTS $temp          # Prints: temporary data

UNSET temp
# PUTS $temp        # ERROR: variable doesn't exist
\end{lstlisting}
\end{examplebox}

\begin{warningbox}
Accessing an unset variable causes an error. Use \cmd{INFO EXISTS} to check if a variable exists before using it.
\end{warningbox}

\section{Variable Expansion}

Variable expansion means replacing \var{varname} with the variable's value.

\subsection{Basic Expansion}

\begin{examplebox}[title=Variable Expansion]
\begin{lstlisting}[language=BCL]
SET fruit "apple"
SET count 5

PUTS "I have $count ${fruit}s"  # Note: BCL uses only $var
PUTS "I have $count apples"
\end{lstlisting}

\textbf{Output:}
\begin{verbatim}
I have 5 apples
I have 5 apples
\end{verbatim}
\end{examplebox}

\begin{notebox}
BCL uses only the \texttt{\$var} form for variable expansion. Unlike Tcl, \texttt{\$\{var\}} is not supported.
\end{notebox}

\subsection{String Concatenation}

You can concatenate strings by placing variable expansions next to each other:

\begin{examplebox}[title=Concatenation]
\begin{lstlisting}[language=BCL]
SET first "John"
SET last "Doe"
SET fullname $first" "$last
PUTS $fullname  # Prints: John Doe

# Alternative with variables only
SET a "Hello"
SET b "World"
SET c $a$b
PUTS $c  # Prints: HelloWorld
\end{lstlisting}
\end{examplebox}

\section{INCR: Incrementing Numbers}

\cmd{INCR} is a specialized command for incrementing (or decrementing) numeric variables.

\textbf{Syntax:}
\begin{lstlisting}[language=BCL]
INCR varname          # Increment by 1
INCR varname amount   # Increment by amount
\end{lstlisting}

\begin{examplebox}[title=INCR Examples]
\begin{lstlisting}[language=BCL]
SET counter 10

INCR counter       # counter is now 11
INCR counter 5     # counter is now 16
INCR counter -3    # counter is now 13 (decrement)

PUTS "Counter: $counter"
\end{lstlisting}

\textbf{Output:}
\begin{verbatim}
Counter: 13
\end{verbatim}
\end{examplebox}

\section{APPEND: Building Strings}

\cmd{APPEND} adds text to the end of a variable, modifying it in place.

\textbf{Syntax:}
\begin{lstlisting}[language=BCL]
APPEND varname value1 value2 ...
\end{lstlisting}

\begin{examplebox}[title=APPEND Examples]
\begin{lstlisting}[language=BCL]
SET message "Hello"
APPEND message " " "World" "!"
PUTS $message  # Prints: Hello World!

# Building strings in a loop
SET result ""
SET i 1
WHILE $i <= 5 DO
  APPEND result $i " "
  INCR i
END
PUTS $result  # Prints: 1 2 3 4 5
\end{lstlisting}
\end{examplebox}

\begin{tipbox}
\cmd{APPEND} is more efficient than repeated \cmd{SET} operations when building large strings in loops.
\end{tipbox}

\section{Variable Scope}

\subsection{Global Variables}

Variables created outside procedures are global—they can be accessed anywhere in your program.

\begin{lstlisting}[language=BCL]
SET global_var "I am global"

PROC test DO
  # Cannot access global_var here without GLOBAL declaration
END
\end{lstlisting}

\subsection{Local Variables}

Variables created with \cmd{SET} inside a procedure are local—they only exist within that procedure.

\begin{examplebox}[title=Local vs Global]
\begin{lstlisting}[language=BCL]
SET outside "global"

PROC demo DO
  SET inside "local"
  PUTS "Inside proc: $inside"
END

demo
PUTS "Outside proc: $outside"
# PUTS $inside  # ERROR: inside doesn't exist here
\end{lstlisting}

\textbf{Output:}
\begin{verbatim}
Inside proc: local
Outside proc: global
\end{verbatim}
\end{examplebox}

\subsection{The GLOBAL Command}

To access or modify global variables from within a procedure, use \cmd{GLOBAL}.

\begin{examplebox}[title=Using GLOBAL]
\begin{lstlisting}[language=BCL]
SET score 0

PROC add_points WITH points DO
  GLOBAL score
  INCR score $points
END

add_points 10
add_points 5
PUTS "Total score: $score"  # Prints: Total score: 15
\end{lstlisting}
\end{examplebox}

\section{Practical Examples}

\subsection{Counter Example}

\begin{examplebox}[title=Hit Counter]
\begin{lstlisting}[language=BCL]
SET hits 0

PROC record_hit DO
  GLOBAL hits
  INCR hits
  PUTS "Hit number $hits recorded"
END

record_hit
record_hit
record_hit
PUTS "Total hits: $hits"
\end{lstlisting}

\textbf{Output:}
\begin{verbatim}
Hit number 1 recorded
Hit number 2 recorded
Hit number 3 recorded
Total hits: 3
\end{verbatim}
\end{examplebox}

\subsection{Text Builder Example}

\begin{examplebox}[title=Building a Report]
\begin{lstlisting}[language=BCL]
SET report ""
APPEND report "=== SYSTEM REPORT ===\n"
APPEND report "Date: " [CLOCK FORMAT [CLOCK SECONDS]] "\n"
APPEND report "Status: OK\n"
APPEND report "===================="

PUTS $report
\end{lstlisting}
\end{examplebox}
