\label{ch:files}

BCL provides comprehensive file I/O capabilities.

\section{Opening and Closing Files}

\begin{table}[h]
\centering
\caption{File Open Modes}
\label{tab:file_modes}
\begin{tabular}{ll}
\toprule
\textbf{Mode} & \textbf{Description} \\
\midrule
\texttt{R} & Read (file must exist) \\
\texttt{W} & Write (creates or truncates) \\
\texttt{A} & Append (creates if needed) \\
\texttt{RW} & Read and write \\
\bottomrule
\end{tabular}
\end{table}

\begin{examplebox}[title=Basic File I/O]
\begin{lstlisting}[language=BCL]
# Write to file
SET fh [OPEN "output.txt" W]
PUTS $fh "Line 1"
PUTS $fh "Line 2"
CLOSE $fh

# Read from file
SET fh [OPEN "output.txt" R]
SET content [READ $fh]
CLOSE $fh

PUTS "File content:"
PUTS $content
\end{lstlisting}
\end{examplebox}

\section{Reading Files Line by Line}

\begin{examplebox}[title=Line-by-Line Reading]
\begin{lstlisting}[language=BCL]
SET fh [OPEN "data.txt" R]

SET linenum 0
WHILE [EOF $fh] = 0 DO
  SET line [GETS $fh]
  INCR linenum

  IF [STRING LENGTH $line] > 0 THEN
    PUTS "Line $linenum: $line"
  END
END

CLOSE $fh
\end{lstlisting}
\end{examplebox}

\section{File Commands}

\begin{examplebox}[title=FILE Commands]
\begin{lstlisting}[language=BCL]
SET filename "test.txt"

# Check existence
IF [FILE EXISTS $filename] THEN
  PUTS "File exists"

  # Get size
  SET size [FILE SIZE $filename]
  PUTS "Size: $size bytes"

  # Rename
  FILE RENAME $filename "test_backup.txt"

  # Delete
  # FILE DELETE "test_backup.txt"
END

# Get current directory
SET cwd [PWD]
PUTS "Current directory: $cwd"

# Find files matching pattern
SET txtfiles [GLOB "*.txt"]
PUTS "Text files: $txtfiles"
\end{lstlisting}
\end{examplebox}

\section{File Positioning with SEEK}

The \texttt{SEEK} command allows you to move the file pointer to a specific position within a file.

\begin{syntaxbox}
SEEK handle offset whence
\end{syntaxbox}

\begin{itemize}
\item \textbf{handle} -- File handle from \texttt{OPEN}
\item \textbf{offset} -- Byte offset (can be negative for \texttt{END})
\item \textbf{whence} -- Reference point:
  \begin{itemize}
  \item \texttt{START} or \texttt{SET} -- From beginning of file
  \item \texttt{CUR} or \texttt{CURRENT} -- From current position
  \item \texttt{END} -- From end of file
  \end{itemize}
\end{itemize}

\begin{examplebox}[title=Using SEEK]
\begin{lstlisting}[language=BCL]
# Open file for reading
SET fh [OPEN "data.bin" R]

# Read first 10 bytes
SET header [READ $fh 10]

# Go back to start
SEEK $fh 0 START

# Skip 100 bytes from current position
SEEK $fh 100 CUR

# Go to 10 bytes before end
SEEK $fh -10 END

CLOSE $fh
\end{lstlisting}
\end{examplebox}

