% ============================================================================
% CHAPTER 10: REGULAR EXPRESSIONS
% ============================================================================

\chapter{Regular Expressions}
\label{ch:regexp}

Regular expressions (often called "regex" or "regexp") are powerful patterns used to search, match, and manipulate text. Think of them as a sophisticated "find and replace" tool that can match complex patterns instead of just exact text.

\section{What Are Regular Expressions?}

Imagine you want to find all phone numbers in a document, or validate that an email address is correctly formatted, or replace all dates from one format to another. Regular expressions make these tasks easy.

\begin{examplebox}[title=Real-World Analogy]
Regular expressions are like describing something without knowing its exact form:
\begin{itemize}
  \item "Any word starting with 'cat'" - matches "cat", "cats", "category"
  \item "A sequence of digits" - matches "123", "4567", "999"
  \item "Text between quotes" - matches "hello", "goodbye"
\end{itemize}
\end{examplebox}

\begin{notebox}
If you're new to programming, regular expressions might seem cryptic at first. Don't worry! We'll start with simple patterns and build up to more complex ones.
\end{notebox}

\section{Basic Pattern Building Blocks}

Regular expressions are built from simple pieces. Let's learn them one at a time.

\subsection{Literal Characters}

The simplest pattern is just normal text—it matches exactly what you write.

\begin{examplebox}[title=Literal Matching]
\begin{lstlisting}[language=BCL]
SET text "The cat sat on the mat"

# Match the word "cat"
IF [REGEXP "cat" $text] THEN
  PUTS "Found 'cat' in the text"
END

# Match "mat"
IF [REGEXP "mat" $text] THEN
  PUTS "Found 'mat' in the text"
END

# This won't match because case matters
IF [REGEXP "Cat" $text] THEN
  PUTS "Found 'Cat'"
ELSE
  PUTS "'Cat' not found (wrong case)"
END
\end{lstlisting}

\textbf{Output:}
\begin{verbatim}
Found 'cat' in the text
Found 'mat' in the text
'Cat' not found (wrong case)
\end{verbatim}
\end{examplebox}

\subsection{Special Characters - The Wildcards}

Some characters have special meanings in regular expressions:

\begin{table}[h]
\centering
\caption{Basic Regular Expression Characters}
\label{tab:regex_basics}
\begin{tabular}{lp{8cm}}
\toprule
\textbf{Symbol} & \textbf{Meaning} \\
\midrule
\texttt{.} & Matches any single character \\
\texttt{*} & Matches 0 or more of the previous character \\
\texttt{+} & Matches 1 or more of the previous character \\
\texttt{?} & Matches 0 or 1 of the previous character \\
\texttt{\^{}} & Matches the start of the string \\
\texttt{\$} & Matches the end of the string \\
\texttt{|} & OR operator (matches left or right) \\
\texttt{(...)} & Groups patterns together \\
\texttt{[...]} & Matches any character in the brackets \\
\texttt{\textbackslash} & Escapes special characters \\
\bottomrule
\end{tabular}
\end{table}

\subsection{The Dot (.) - Any Character}

The dot matches any single character except newline.

\begin{examplebox}[title=Using the Dot]
\begin{lstlisting}[language=BCL]
# Match "c.t" - c, any character, then t
SET words [LIST "cat" "cot" "cut" "cart" "ct"]

FOREACH word IN $words DO
  IF [REGEXP "c.t" $word] THEN
    PUTS "$word matches c.t"
  ELSE
    PUTS "$word does NOT match c.t"
  END
END
\end{lstlisting}

\textbf{Output:}
\begin{verbatim}
cat matches c.t
cot matches c.t
cut matches c.t
cart matches c.t
ct does NOT match c.t
\end{verbatim}

\textbf{Explanation:} The pattern \texttt{c.t} requires exactly one character between 'c' and 't'. "cart" matches because it contains "car" + "t" = "cart" which has the pattern.
\end{examplebox}

\subsection{Character Classes [...]}

Square brackets match any one character from a set.

\begin{examplebox}[title=Character Classes]
\begin{lstlisting}[language=BCL]
# Match c[aou]t - cat, cot, or cut
SET words [LIST "cat" "cot" "cut" "cit" "cet"]

FOREACH word IN $words DO
  IF [REGEXP "c\[aou\]t" $word] THEN
    PUTS "$word matches"
  END
END

# Match any digit [0-9]
SET text "I have 5 apples"
IF [REGEXP "\[0-9\]" $text] THEN
  PUTS "Contains a digit"
END

# Match any letter [a-z] or [A-Z]
IF [REGEXP "\[a-z\]" $text] THEN
  PUTS "Contains lowercase letters"
END
\end{lstlisting}

\textbf{Output:}
\begin{verbatim}
cat matches
cot matches
cut matches
Contains a digit
Contains lowercase letters
\end{verbatim}
\end{examplebox}

\begin{tipbox}
Common character classes:
\begin{itemize}
  \item \texttt{[0-9]} - any digit
  \item \texttt{[a-z]} - any lowercase letter
  \item \texttt{[A-Z]} - any uppercase letter
  \item \texttt{[a-zA-Z]} - any letter
  \item \texttt{[\^{}0-9]} - anything that's NOT a digit
\end{itemize}
\end{tipbox}

\subsection{Repetition: *, +, ?}

These tell how many times a pattern should repeat.

\begin{table}[h]
\centering
\caption{Repetition Operators}
\label{tab:regex_repetition}
\begin{tabular}{lp{8cm}}
\toprule
\textbf{Operator} & \textbf{Meaning} \\
\midrule
\texttt{*} & 0 or more times (optional, can repeat) \\
\texttt{+} & 1 or more times (must appear at least once) \\
\texttt{?} & 0 or 1 time (optional, appears once or not at all) \\
\texttt{\{n\}} & Exactly n times \\
\texttt{\{n,\}} & n or more times \\
\texttt{\{n,m\}} & Between n and m times \\
\bottomrule
\end{tabular}
\end{table}

\begin{examplebox}[title=Repetition Examples]
\begin{lstlisting}[language=BCL]
# Match one or more digits
SET texts [LIST "abc" "123" "abc123" "a1b2c3"]

FOREACH text IN $texts DO
  IF [REGEXP "\[0-9\]+" $text] THEN
    PUTS "$text contains numbers"
  END
END

# Match optional minus sign followed by digits: -?[0-9]+
SET numbers [LIST "123" "-456" "78" "-9"]
FOREACH num IN $numbers DO
  IF [REGEXP "^-?\[0-9\]+$" $num] THEN
    PUTS "$num is a valid number"
  END
END

# Match "color" or "colour"
IF [REGEXP "colou?r" "color"] THEN
  PUTS "Matches 'color'"
END
IF [REGEXP "colou?r" "colour"] THEN
  PUTS "Matches 'colour'"
END
\end{lstlisting}

\textbf{Output:}
\begin{verbatim}
123 contains numbers
abc123 contains numbers
a1b2c3 contains numbers
123 is a valid number
-456 is a valid number
78 is a valid number
-9 is a valid number
Matches 'color'
Matches 'colour'
\end{verbatim}
\end{examplebox}

\subsection{Anchors: \^{} and \$}

Anchors match positions, not characters.

\begin{itemize}
  \item \texttt{\^{}} matches the start of the string
  \item \texttt{\$} matches the end of the string
\end{itemize}

\begin{examplebox}[title=Using Anchors]
\begin{lstlisting}[language=BCL]
SET text "hello world"

# Must start with "hello"
IF [REGEXP "^hello" $text] THEN
  PUTS "Starts with 'hello'"
END

# Must end with "world"
IF [REGEXP "world$" $text] THEN
  PUTS "Ends with 'world'"
END

# Must be EXACTLY "hello world" (nothing before or after)
IF [REGEXP "^hello world$" $text] THEN
  PUTS "Exact match"
END

# Won't match - has extra text
SET text2 "say hello world now"
IF [REGEXP "^hello world$" $text2] THEN
  PUTS "Exact match"
ELSE
  PUTS "Not an exact match - has extra text"
END
\end{lstlisting}

\textbf{Output:}
\begin{verbatim}
Starts with 'hello'
Ends with 'world'
Exact match
Not an exact match - has extra text
\end{verbatim}
\end{examplebox}

\subsection{Shorthand Character Classes}

BCL provides shortcuts for common patterns:

\begin{table}[h]
\centering
\caption{Shorthand Character Classes}
\label{tab:regex_shorthand}
\begin{tabular}{lp{8cm}}
\toprule
\textbf{Shorthand} & \textbf{Equivalent To} \\
\midrule
\texttt{\textbackslash d} & \texttt{[0-9]} - any digit \\
\texttt{\textbackslash D} & \texttt{[\^{}0-9]} - any non-digit \\
\texttt{\textbackslash w} & \texttt{[a-zA-Z0-9\_]} - word character \\
\texttt{\textbackslash W} & Non-word character \\
\texttt{\textbackslash s} & Whitespace (space, tab, newline) \\
\texttt{\textbackslash S} & Non-whitespace \\
\bottomrule
\end{tabular}
\end{table}

\begin{examplebox}[title=Shorthand Examples]
\begin{lstlisting}[language=BCL]
# Find digits with \d
SET text "Room 101 is on floor 5"
IF [REGEXP "\\d+" $text] THEN
  PUTS "Found numbers"
END

# Find words with \w
IF [REGEXP "\\w+" $text] THEN
  PUTS "Found word characters"
END

# Validate format: word space word
SET input "hello world"
IF [REGEXP "^\\w+\\s+\\w+$" $input] THEN
  PUTS "Valid format: two words separated by space"
END
\end{lstlisting}

\textbf{Output:}
\begin{verbatim}
Found numbers
Found word characters
Valid format: two words separated by space
\end{verbatim}
\end{examplebox}

\begin{warningbox}
In BCL, you need to escape backslashes in strings. Write \texttt{\textbackslash\textbackslash d} to represent \texttt{\textbackslash d} in the pattern.
\end{warningbox}

\section{The REGEXP Command}

\cmd{REGEXP} checks if a pattern matches and can extract matched portions.

\textbf{Syntax:}
\begin{lstlisting}[language=BCL]
REGEXP pattern string                    # Returns 1 if match, 0 if not
REGEXP pattern string matchvar           # Store entire match
REGEXP pattern string matchvar subvar... # Store submatches
\end{lstlisting}

\subsection{Basic Matching}

\begin{examplebox}[title=Simple Pattern Matching]
\begin{lstlisting}[language=BCL]
SET email "user@example.com"

# Check if it looks like an email
IF [REGEXP "@" $email] THEN
  PUTS "Contains @ symbol"
END

# Check for email pattern: word @ word . word
IF [REGEXP "\\w+@\\w+\\.\\w+" $email] THEN
  PUTS "Looks like a valid email"
END

# Validate phone number: exactly 10 digits
SET phone "5551234567"
IF [REGEXP "^\\d\{10\}$" $phone] THEN
  PUTS "Valid 10-digit phone number"
END
\end{lstlisting}

\textbf{Output:}
\begin{verbatim}
Contains @ symbol
Looks like a valid email
Valid 10-digit phone number
\end{verbatim}
\end{examplebox}

\subsection{Capturing Matches}

Use parentheses \texttt{(...)} to capture parts of the match.

\begin{examplebox}[title=Extracting Information]
\begin{lstlisting}[language=BCL]
# Extract year from date
SET date "Today is 2025-10-22"
REGEXP "(\\d\{4\})-(\\d\{2\})-(\\d\{2\})" $date MATCH year month day
PUTS "Year: $year"
PUTS "Month: $month"
PUTS "Day: $day"

# Extract name and extension from filename
SET filename "document.pdf"
REGEXP "(.+)\\.(\\w+)$" $filename MATCH name ext
PUTS "Name: $name"
PUTS "Extension: $ext"

# Extract email parts
SET email "john.doe@example.com"
REGEXP "(.+)@(.+)" $email MATCH username domain
PUTS "Username: $username"
PUTS "Domain: $domain"
\end{lstlisting}

\textbf{Output:}
\begin{verbatim}
Year: 2025
Month: 10
Day: 22
Name: document
Extension: pdf
Username: john.doe
Domain: example.com
\end{verbatim}
\end{examplebox}

\subsection{REGEXP Options}

\begin{table}[h]
\centering
\caption{REGEXP Options}
\label{tab:regexp_options}
\begin{tabular}{lp{8cm}}
\toprule
\textbf{Option} & \textbf{Description} \\
\midrule
\texttt{-nocase} & Case-insensitive matching \\
\texttt{-line} & Treat string as multiple lines (\^{} and \$ match line starts/ends) \\
\texttt{-lineanchor} & Similar to -line \\
\texttt{-expanded} & Ignore whitespace in pattern (for readability) \\
\bottomrule
\end{tabular}
\end{table}

\begin{examplebox}[title=Case-Insensitive Matching]
\begin{lstlisting}[language=BCL]
SET text "Hello World"

# Case-sensitive (won't match)
IF [REGEXP "hello" $text] THEN
  PUTS "Found (sensitive)"
ELSE
  PUTS "Not found (sensitive)"
END

# Case-insensitive (will match)
IF [REGEXP -nocase "hello" $text] THEN
  PUTS "Found (insensitive)"
END
\end{lstlisting}

\textbf{Output:}
\begin{verbatim}
Not found (sensitive)
Found (insensitive)
\end{verbatim}
\end{examplebox}

\section{The REGSUB Command}

\cmd{REGSUB} replaces text that matches a pattern.

\textbf{Syntax:}
\begin{lstlisting}[language=BCL]
REGSUB pattern string replacement         # Replace first match
REGSUB pattern string replacement ALL     # Replace all matches
\end{lstlisting}

\subsection{Basic Replacement}

\begin{examplebox}[title=Simple Replacements]
\begin{lstlisting}[language=BCL]
SET text "Hello, World!"

# Replace first occurrence
SET result [REGSUB "World" $text "BCL"]
PUTS $result  # Hello, BCL!

# Replace all occurrences
SET text2 "foo bar foo baz foo"
SET result2 [REGSUB "foo" $text2 "XXX" ALL]
PUTS $result2  # XXX bar XXX baz XXX

# Remove all digits
SET text3 "Room 101 is on floor 5"
SET clean [REGSUB "\\d+" $text3 "" ALL]
PUTS $clean  # Room  is on floor
\end{lstlisting}

\textbf{Output:}
\begin{verbatim}
Hello, BCL!
XXX bar XXX baz XXX
Room  is on floor
\end{verbatim}
\end{examplebox}

\subsection{Using Captured Groups in Replacement}

You can reference captured groups in the replacement using \texttt{\&} or \texttt{\textbackslash 1}, \texttt{\textbackslash 2}, etc.

\begin{examplebox}[title=Advanced Replacements]
\begin{lstlisting}[language=BCL]
# Swap first and last name
SET name "John Doe"
SET swapped [REGSUB "(\\w+) (\\w+)" $name "\\2, \\1"]
PUTS $swapped  # Doe, John

# Format phone number: 5551234567 -> (555) 123-4567
SET phone "5551234567"
SET formatted [REGSUB "(\\d\{3\})(\\d\{3\})(\\d\{4\})" $phone "(\\1) \\2-\\3"]
PUTS $formatted  # (555) 123-4567

# Add "http://" to URLs that don't have it
SET url "example.com"
IF [REGEXP "^http" $url] = 0 THEN
  SET url [REGSUB "^" $url "http://"]
END
PUTS $url  # http://example.com
\end{lstlisting}

\textbf{Output:}
\begin{verbatim}
Doe, John
(555) 123-4567
http://example.com
\end{verbatim}
\end{examplebox}

\section{Practical Examples}

\subsection{Email Validator (Advanced)}

\begin{examplebox}[title=Complete Email Validation]
\begin{lstlisting}[language=BCL]
PROC validate_email WITH email DO
  # Basic pattern: user@domain.tld
  SET pattern "^\\w+(\\.\\w+)*@\\w+(\\.\\w+)+$"

  IF [REGEXP $pattern $email] THEN
    PUTS "$email is VALID"
    RETURN 1
  ELSE
    PUTS "$email is INVALID"
    RETURN 0
  END
END

# Test cases
validate_email "user@example.com"
validate_email "john.doe@company.co.uk"
validate_email "invalid@"
validate_email "@invalid.com"
validate_email "no-at-sign.com"
\end{lstlisting}

\textbf{Output:}
\begin{verbatim}
user@example.com is VALID
john.doe@company.co.uk is VALID
invalid@ is INVALID
@invalid.com is INVALID
no-at-sign.com is INVALID
\end{verbatim}
\end{examplebox}

\subsection{Extract URLs from Text}

\begin{examplebox}[title=Finding URLs]
\begin{lstlisting}[language=BCL]
PROC extract_urls WITH text DO
  SET urls [LIST]

  # Pattern for http(s) URLs
  SET pattern "https?://\[\\w\\.-\]+\[/\\w\\.-\]*"

  SET pos 0
  WHILE 1 DO
    # Find next URL
    IF [REGEXP -start $pos $pattern $text url] THEN
      SET urls [LAPPEND $urls $url]
      # Move past this match
      SET pos [STRING FIRST $url $text $pos]
      INCR pos [STRING LENGTH $url]
    ELSE
      BREAK
    END
  END

  RETURN $urls
END

SET text "Visit http://example.com or https://bcl.org for info"
SET urls [extract_urls $text]
PUTS "Found URLs:"
FOREACH url IN $urls DO
  PUTS "  - $url"
END
\end{lstlisting}

\textbf{Output:}
\begin{verbatim}
Found URLs:
  - http://example.com
  - https://bcl.org
\end{verbatim}
\end{examplebox}

\subsection{Password Strength Validator}

\begin{examplebox}[title=Regex-based Password Checking]
\begin{lstlisting}[language=BCL]
PROC check_password_strength WITH password DO
  SET score 0

  # Check length
  IF [STRING LENGTH $password] >= 8 THEN
    INCR score
  END

  # Check for lowercase
  IF [REGEXP "\[a-z\]" $password] THEN
    INCR score
  END

  # Check for uppercase
  IF [REGEXP "\[A-Z\]" $password] THEN
    INCR score
  END

  # Check for digits
  IF [REGEXP "\\d" $password] THEN
    INCR score
  END

  # Check for special characters
  IF [REGEXP "\[!@#$%^&*\]" $password] THEN
    INCR score
  END

  # Report strength
  IF $score < 3 THEN
    RETURN "Weak"
  ELSEIF $score < 5 THEN
    RETURN "Medium"
  ELSE
    RETURN "Strong"
  END
END

SET passwords [LIST "hello" "Hello123" "MyP@ss123" "Complex$Pass9"]
FOREACH pass IN $passwords DO
  SET strength [check_password_strength $pass]
  PUTS "$pass: $strength"
END
\end{lstlisting}

\textbf{Output:}
\begin{verbatim}
hello: Weak
Hello123: Medium
MyP@ss123: Strong
Complex$Pass9: Strong
\end{verbatim}
\end{examplebox}

\subsection{Format Phone Numbers}

\begin{examplebox}[title=Normalize Phone Numbers]
\begin{lstlisting}[language=BCL]
PROC format_phone WITH phone DO
  # Remove all non-digits
  SET clean [REGSUB "\\D" $phone "" ALL]

  # Check if we have 10 digits
  IF [STRING LENGTH $clean] != 10 THEN
    RETURN "Invalid phone number"
  END

  # Format as (XXX) XXX-XXXX
  SET formatted [REGSUB "(\\d\{3\})(\\d\{3\})(\\d\{4\})" $clean "(\\1) \\2-\\3"]
  RETURN $formatted
END

# Test with various formats
SET phones [LIST "5551234567" "555-123-4567" "(555) 123-4567" "555.123.4567"]
FOREACH phone IN $phones DO
  SET formatted [format_phone $phone]
  PUTS "$phone -> $formatted"
END
\end{lstlisting}

\textbf{Output:}
\begin{verbatim}
5551234567 -> (555) 123-4567
555-123-4567 -> (555) 123-4567
(555) 123-4567 -> (555) 123-4567
555.123.4567 -> (555) 123-4567
\end{verbatim}
\end{examplebox}

\section{Common Regular Expression Patterns}

Here's a reference of useful patterns for common tasks:

\begin{table}[h]
\centering
\caption{Common Regex Patterns}
\label{tab:common_patterns}
\begin{tabular}{lp{7cm}}
\toprule
\textbf{Pattern} & \textbf{Description} \\
\midrule
\texttt{\^{}\textbackslash d\{4\}-\textbackslash d\{2\}-\textbackslash d\{2\}\$} & Date (YYYY-MM-DD) \\
\texttt{\textbackslash w+@\textbackslash w+\textbackslash .\textbackslash w+} & Simple email \\
\texttt{\^{}\textbackslash d\{3\}-\textbackslash d\{3\}-\textbackslash d\{4\}\$} & Phone (XXX-XXX-XXXX) \\
\texttt{\^{}[01]?\textbackslash d\textbackslash d?\$} & Number 0-199 \\
\texttt{\textbackslash b\textbackslash w\{3\}\textbackslash b} & Exactly 3-letter word \\
\texttt{https?://.*} & HTTP or HTTPS URL \\
\texttt{\^{}\textbackslash s*\$} & Empty or whitespace only \\
\texttt{\textbackslash d+\textbackslash .\textbackslash d\{2\}} & Decimal with 2 places \\
\bottomrule
\end{tabular}
\end{table}

\begin{tipbox}
Regular expressions can get complex. Start simple and build up gradually. Test your patterns with various inputs to ensure they work as expected.
\end{tipbox}

\begin{warningbox}
Be careful with patterns like \texttt{.*} (match anything) as they can match more than you expect. Use specific patterns when possible.
\end{warningbox}
