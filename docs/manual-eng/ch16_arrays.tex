% ============================================================================
% CHAPTER 16: ASSOCIATIVE ARRAYS
% ============================================================================

\chapter{Associative Arrays}
\label{ch:arrays}

BCL supports associative arrays (also known as dictionaries or hash maps in other languages), inspired by Tcl's array syntax. Arrays allow you to store values indexed by arbitrary keys—either numbers or strings.

\section{Basic Syntax}

\textbf{Assignment:}
\begin{lstlisting}[language=BCL]
SET arrayName(index) value
\end{lstlisting}

\textbf{Access:}
\begin{lstlisting}[language=BCL]
$arrayName(index)
\end{lstlisting}

\begin{examplebox}[title=Simple Array]
\begin{lstlisting}[language=BCL]
SET fruits(1) "apple"
SET fruits(2) "orange"
SET fruits(3) "banana"

PUTS $fruits(1)    # apple
PUTS $fruits(2)    # orange
PUTS $fruits(3)    # banana
\end{lstlisting}

\textbf{Output:}
\begin{verbatim}
apple
orange
banana
\end{verbatim}
\end{examplebox}

\section{Numeric Indices}

Arrays can use numbers as indices, similar to traditional arrays:

\begin{examplebox}[title=Numeric Indices with Loop]
\begin{lstlisting}[language=BCL]
SET colors(1) "red"
SET colors(2) "green"
SET colors(3) "blue"

SET i 1
WHILE $i <= 3 DO
  PUTS "colors($i) = $colors($i)"
  INCR i
END
\end{lstlisting}

\textbf{Output:}
\begin{verbatim}
colors(1) = red
colors(2) = green
colors(3) = blue
\end{verbatim}
\end{examplebox}

\section{String Indices (Associative)}

The real power of BCL arrays comes from using string keys:

\begin{examplebox}[title=Phone Directory]
\begin{lstlisting}[language=BCL]
SET phone(John) "555-1234"
SET phone(Mary) "555-5678"
SET phone(Peter) "555-9012"

SET contact "Mary"
PUTS "Phone for $contact: $phone($contact)"
\end{lstlisting}

\textbf{Output:}
\begin{verbatim}
Phone for Mary: 555-5678
\end{verbatim}
\end{examplebox}

\section{Variable Indices}

You can use variables as array indices:

\begin{examplebox}[title=Variable as Index]
\begin{lstlisting}[language=BCL]
SET data(Monday) "10"
SET data(Tuesday) "15"
SET data(Wednesday) "12"

SET day "Tuesday"
PUTS "Data for $day: $data($day)"
\end{lstlisting}

\textbf{Output:}
\begin{verbatim}
Data for Tuesday: 15
\end{verbatim}
\end{examplebox}

\section{Expression Indices}

Indices can be expressions, allowing for calculated array access:

\begin{examplebox}[title=Expression as Index]
\begin{lstlisting}[language=BCL]
SET table(1) "A"
SET table(2) "B"
SET table(3) "C"

SET i 1
SET j [EXPR $i + 1]
PUTS $table($j)    # B

SET k [EXPR $i * 3]
PUTS $table($k)    # C
\end{lstlisting}

\textbf{Output:}
\begin{verbatim}
B
C
\end{verbatim}
\end{examplebox}

\section{Multidimensional Arrays (Simulated)}

BCL simulates multidimensional arrays using composite indices:

\begin{examplebox}[title=2D Matrix]
\begin{lstlisting}[language=BCL]
SET matrix(1,1) "A"
SET matrix(1,2) "B"
SET matrix(2,1) "C"
SET matrix(2,2) "D"

PUTS "$matrix(1,1) $matrix(1,2)"
PUTS "$matrix(2,1) $matrix(2,2)"
\end{lstlisting}

\textbf{Output:}
\begin{verbatim}
A B
C D
\end{verbatim}
\end{examplebox}

\section{Checking Element Existence}

Use \cmd{INFO EXISTS} to check if an array element exists:

\begin{examplebox}[title=Check Existence]
\begin{lstlisting}[language=BCL]
SET config(debug) "true"

IF [INFO EXISTS config(debug)] THEN
  PUTS "Debug is: $config(debug)"
ELSE
  PUTS "Debug not set"
END

IF [INFO EXISTS config(timeout)] THEN
  PUTS "Timeout is: $config(timeout)"
ELSE
  PUTS "Timeout not configured"
END
\end{lstlisting}

\textbf{Output:}
\begin{verbatim}
Debug is: true
Timeout not configured
\end{verbatim}
\end{examplebox}

\section{Practical Examples}

\subsection{Configuration Settings}

\begin{examplebox}[title=Application Config]
\begin{lstlisting}[language=BCL]
SET config(host) "localhost"
SET config(port) "8080"
SET config(timeout) "30"
SET config(debug) "false"

PUTS "Server: $config(host):$config(port)"
PUTS "Timeout: $config(timeout)s"
PUTS "Debug mode: $config(debug)"
\end{lstlisting}

\textbf{Output:}
\begin{verbatim}
Server: localhost:8080
Timeout: 30s
Debug mode: false
\end{verbatim}
\end{examplebox}

\subsection{Event Counter}

\begin{examplebox}[title=Event Tracking]
\begin{lstlisting}[language=BCL]
SET counter(login) "0"
SET counter(logout) "0"
SET counter(error) "0"

# Simulate events
SET counter(login) [EXPR $counter(login) + 1]
SET counter(login) [EXPR $counter(login) + 1]
SET counter(logout) [EXPR $counter(logout) + 1]
SET counter(error) [EXPR $counter(error) + 1]
SET counter(login) [EXPR $counter(login) + 1]

PUTS "Login: $counter(login)"
PUTS "Logout: $counter(logout)"
PUTS "Error: $counter(error)"
\end{lstlisting}

\textbf{Output:}
\begin{verbatim}
Login: 3
Logout: 1
Error: 1
\end{verbatim}
\end{examplebox}

\subsection{Multiplication Table}

\begin{examplebox}[title=Times Table]
\begin{lstlisting}[language=BCL]
SET num 5
SET i 1
WHILE $i <= 10 DO
  SET table($num,$i) [EXPR $num * $i]
  PUTS "$num x $i = $table($num,$i)"
  INCR i
END
\end{lstlisting}

\textbf{Output:}
\begin{verbatim}
5 x 1 = 5
5 x 2 = 10
...
5 x 10 = 50
\end{verbatim}
\end{examplebox}

\section{Arrays vs Lists}

\begin{tipbox}
\textbf{When to use Arrays:}
\begin{itemize}
  \item Key-value mappings (phone directory, configuration)
  \item Arbitrary indices (non-consecutive numbers or strings)
  \item Fast lookup by key
\end{itemize}

\textbf{When to use Lists:}
\begin{itemize}
  \item Ordered collections
  \item Sequential processing
  \item Operations like sorting, joining, splitting
\end{itemize}
\end{tipbox}

\section{Arrays in Procedures}

Arrays can be used with the \cmd{GLOBAL} keyword in procedures:

\begin{examplebox}[title=Global Arrays in Procedures]
\begin{lstlisting}[language=BCL]
PROC show_person DO
  GLOBAL person
  PUTS "Name: $person(name)"
  PUTS "Age: $person(age)"
END

SET person(name) "Alice"
SET person(age) "30"
show_person
\end{lstlisting}

\textbf{Output:}
\begin{verbatim}
Name: Alice
Age: 30
\end{verbatim}
\end{examplebox}

\begin{notebox}
Array elements are stored as individual variables with names like \texttt{arrayname(index)}. They behave like regular variables and follow the same scoping rules.
\end{notebox}

\section{The ARRAY Command}

BCL provides the \cmd{ARRAY} command for advanced array manipulation, inspired by Tcl. This command offers powerful operations for working with arrays as complete structures.

\subsection{ARRAY EXISTS}

Check if an array exists (has at least one element):

\begin{lstlisting}[language=BCL]
ARRAY EXISTS arrayName
\end{lstlisting}

Returns \texttt{"1"} if the array exists, \texttt{"0"} otherwise.

\begin{examplebox}[title=Checking Array Existence]
\begin{lstlisting}[language=BCL]
SET result [ARRAY EXISTS config]    # "0"
SET config(debug) "true"
SET result [ARRAY EXISTS config]    # "1"
\end{lstlisting}
\end{examplebox}

\subsection{ARRAY SIZE}

Get the number of elements in an array:

\begin{lstlisting}[language=BCL]
ARRAY SIZE arrayName
\end{lstlisting}

\begin{examplebox}[title=Array Size]
\begin{lstlisting}[language=BCL]
SET colors(red) "#FF0000"
SET colors(green) "#00FF00"
SET colors(blue) "#0000FF"

SET size [ARRAY SIZE colors]       # "3"
PUTS "Array has $size elements"
\end{lstlisting}

\textbf{Output:}
\begin{verbatim}
Array has 3 elements
\end{verbatim}
\end{examplebox}

\subsection{ARRAY NAMES}

Get a list of array indices, optionally filtered by a pattern:

\begin{lstlisting}[language=BCL]
ARRAY NAMES arrayName ?pattern?
\end{lstlisting}

Patterns support glob wildcards: \texttt{*} (any characters), \texttt{?} (one character), \texttt{[abc]} (character set).

\begin{examplebox}[title=Listing Array Indices]
\begin{lstlisting}[language=BCL]
SET data(name) "John"
SET data(age) "30"
SET data(grade1) "8"
SET data(grade2) "9"

SET all [ARRAY NAMES data]            # All indices
SET grades [ARRAY NAMES data "grade*"] # Only grades
SET with_a [ARRAY NAMES data "*a*"]    # Containing 'a'

PUTS "All: $all"
PUTS "Grades: $grades"
PUTS "With 'a': $with_a"
\end{lstlisting}

\textbf{Output:}
\begin{verbatim}
All: name age grade1 grade2
Grades: grade1 grade2
With 'a': name age grade1 grade2
\end{verbatim}
\end{examplebox}

\subsection{ARRAY GET}

Get array contents as a list of index-value pairs:

\begin{lstlisting}[language=BCL]
ARRAY GET arrayName ?pattern?
\end{lstlisting}

Returns alternating indices and values, optionally filtered by pattern.

\begin{examplebox}[title=Getting Array Contents]
\begin{lstlisting}[language=BCL]
SET colors(red) "255,0,0"
SET colors(green) "0,255,0"
SET colors(blue) "0,0,255"

SET all [ARRAY GET colors]
PUTS "All colors: $all"
\end{lstlisting}

\textbf{Output:}
\begin{verbatim}
All colors: red 255,0,0 green 0,255,0 blue 0,0,255
\end{verbatim}
\end{examplebox}

\subsection{ARRAY SET}

Populate an array from a list of index-value pairs:

\begin{lstlisting}[language=BCL]
ARRAY SET arrayName list
\end{lstlisting}

The list must have an even number of elements.

\begin{examplebox}[title=Setting Array from List]
\begin{lstlisting}[language=BCL]
# Create array from list
ARRAY SET config "host localhost port 8080 debug true"

PUTS "Host: $config(host)"
PUTS "Port: $config(port)"
PUTS "Debug: $config(debug)"
\end{lstlisting}

\textbf{Output:}
\begin{verbatim}
Host: localhost
Port: 8080
Debug: true
\end{verbatim}
\end{examplebox}

\textbf{Copying Arrays:}

\begin{examplebox}[title=Copying an Array]
\begin{lstlisting}[language=BCL]
SET original(a) "1"
SET original(b) "2"
SET original(c) "3"

# Copy array
SET data [ARRAY GET original]
ARRAY SET copy $data

PUTS "Copy size: [ARRAY SIZE copy]"
PUTS "copy(a) = $copy(a)"
\end{lstlisting}

\textbf{Output:}
\begin{verbatim}
Copy size: 3
copy(a) = 1
\end{verbatim}
\end{examplebox}

\subsection{ARRAY UNSET}

Delete array elements matching a pattern:

\begin{lstlisting}[language=BCL]
ARRAY UNSET arrayName ?pattern?
\end{lstlisting}

If no pattern is specified, deletes the entire array.

\begin{examplebox}[title=Selective Deletion]
\begin{lstlisting}[language=BCL]
SET cache(temp_1) "data1"
SET cache(temp_2) "data2"
SET cache(perm_1) "data3"

PUTS "Before: [ARRAY SIZE cache]"    # "3"
ARRAY UNSET cache "temp_*"
PUTS "After: [ARRAY SIZE cache]"     # "1"
\end{lstlisting}

\textbf{Output:}
\begin{verbatim}
Before: 3
After: 1
\end{verbatim}
\end{examplebox}

\begin{examplebox}[title=Deleting Entire Array]
\begin{lstlisting}[language=BCL]
SET myarray(1) "value1"
SET myarray(2) "value2"

ARRAY UNSET myarray

SET exists [ARRAY EXISTS myarray]    # "0"
PUTS "Array exists: $exists"
\end{lstlisting}

\textbf{Output:}
\begin{verbatim}
Array exists: 0
\end{verbatim}
\end{examplebox}

\subsection{Practical Example: Configuration Manager}

\begin{examplebox}[title=Configuration Manager]
\begin{lstlisting}[language=BCL]
# Load configuration
ARRAY SET config "
    db_host localhost
    db_port 3306
    db_name myapp
    cache_enabled true
    cache_ttl 300
"

# Show all database settings
PUTS "Database Configuration:"
SET db_keys [ARRAY NAMES config "db_*"]
SET i 0
WHILE $i < [LLENGTH $db_keys] DO
    SET key [LINDEX $db_keys $i]
    PUTS "  $key = $config($key)"
    INCR i
END

# Show cache settings
PUTS ""
PUTS "Cache Configuration:"
SET cache_keys [ARRAY NAMES config "cache_*"]
SET i 0
WHILE $i < [LLENGTH $cache_keys] DO
    SET key [LINDEX $cache_keys $i]
    PUTS "  $key = $config($key)"
    INCR i
END
\end{lstlisting}

\textbf{Output:}
\begin{verbatim}
Database Configuration:
  db_host = localhost
  db_port = 3306
  db_name = myapp

Cache Configuration:
  cache_enabled = true
  cache_ttl = 300
\end{verbatim}
\end{examplebox}
