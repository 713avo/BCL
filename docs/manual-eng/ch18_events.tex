\chapter{Event System}
\label{ch:events}

BCL v2.0 introduces a complete event-driven programming system that enables asynchronous I/O operations and timer-based callbacks. The event system is similar to Tcl's \texttt{fileevent} mechanism and integrates seamlessly with the dynamic extension system.

\section{Overview}

The event system provides a way to register callbacks that are executed when specific events occur:

\begin{itemize}
    \item \textbf{I/O Events}: File descriptors becoming readable, writable, or encountering exceptions
    \item \textbf{Timer Events}: Callbacks fired after a specified delay (milliseconds)
    \item \textbf{Custom Events}: Extension-defined events (GPIO, signals, etc.)
\end{itemize}

\subsection{Architecture}

The BCL event system uses a hybrid architecture:

\begin{itemize}
    \item \textbf{Core}: POSIX \texttt{select()} for portable I/O multiplexing
    \item \textbf{Timers}: High-precision millisecond timers
    \item \textbf{Extensions}: Optional backends (epoll, kqueue) for high performance
\end{itemize}

\section{EVENT Command}

The \texttt{EVENT} command has six subcommands for managing the event loop:

\subsection{EVENT CREATE - Register I/O Event}

\begin{lstlisting}[style=bcl]
EVENT CREATE fd event_type callback_proc
\end{lstlisting}

Register a callback for a file descriptor event.

\textbf{Parameters:}
\begin{itemize}
    \item \texttt{fd} - File descriptor number
    \item \texttt{event\_type} - READABLE, WRITABLE, or EXCEPTION
    \item \texttt{callback\_proc} - Procedure name (receives fd as parameter)
\end{itemize}

\textbf{Example:}
\begin{lstlisting}[style=bcl]
PROC ON_DATA WITH fd DO
    SET data [READ $fd 1024]
    PUTS "Received: $data"
END

EVENT CREATE $socket READABLE ON_DATA
\end{lstlisting}

\subsection{EVENT DELETE - Unregister Event}

\begin{lstlisting}[style=bcl]
EVENT DELETE fd [event_type]
\end{lstlisting}

Remove a registered event handler.

\textbf{Parameters:}
\begin{itemize}
    \item \texttt{fd} - File descriptor
    \item \texttt{event\_type} - (Optional) READABLE, WRITABLE, or EXCEPTION. If omitted, all events for this fd are removed
\end{itemize}

\subsection{EVENT TIMER - Register Timer}

\begin{lstlisting}[style=bcl]
EVENT TIMER milliseconds callback_proc
\end{lstlisting}

Register a one-shot timer callback.

\textbf{Parameters:}
\begin{itemize}
    \item \texttt{milliseconds} - Delay before callback (0 = immediate)
    \item \texttt{callback\_proc} - Procedure name (no parameters)
\end{itemize}

\textbf{Example:}
\begin{lstlisting}[style=bcl]
PROC ON_TIMEOUT DO
    PUTS "5 seconds elapsed"
END

EVENT TIMER 5000 ON_TIMEOUT
\end{lstlisting}

\textbf{Note:} Timers are one-shot. To create repeating timers, re-register the timer within the callback.

\subsection{EVENT PROCESS - Process Once}

\begin{lstlisting}[style=bcl]
EVENT PROCESS [timeout_ms]
\end{lstlisting}

Process pending events once, with optional timeout.

\textbf{Parameters:}
\begin{itemize}
    \item \texttt{timeout\_ms} - Maximum wait time in milliseconds (0 = poll, omit = block indefinitely)
\end{itemize}

\textbf{Returns:} 1 if an event was processed, 0 on timeout

\textbf{Example:}
\begin{lstlisting}[style=bcl]
# Poll for events without blocking
SET result [EVENT PROCESS 0]

# Wait up to 1 second
SET result [EVENT PROCESS 1000]
\end{lstlisting}

\subsection{EVENT LOOP - Run Event Loop}

\begin{lstlisting}[style=bcl]
EVENT LOOP
\end{lstlisting}

Run the event loop indefinitely until \texttt{EVENT STOP} is called.

\textbf{Example:}
\begin{lstlisting}[style=bcl]
# Set up events...
EVENT CREATE $server READABLE ON_ACCEPT
EVENT TIMER 60000 ON_HEARTBEAT

# Run forever
PUTS "Server started"
EVENT LOOP  # Blocks until EVENT STOP
\end{lstlisting}

\subsection{EVENT STOP - Stop Event Loop}

\begin{lstlisting}[style=bcl]
EVENT STOP
\end{lstlisting}

Stop the event loop (causes \texttt{EVENT LOOP} to return).

\subsection{EVENT INFO - List Events}

\begin{lstlisting}[style=bcl]
SET info [EVENT INFO]
\end{lstlisting}

Returns a list of all registered events in human-readable format.

\section{Complete Examples}

\subsection{TCP Echo Server}

\begin{lstlisting}[style=bcl]
LOAD "extensions/socket.so"

GLOBAL clients
SET clients [LIST]

# Handle new connections
PROC ON_ACCEPT WITH server_fd DO
    GLOBAL clients
    SET client [SOCKET ACCEPT $server_fd]
    SET clients [LAPPEND $clients $client]
    PUTS "Client $client connected"
    
    # Register data handler
    EVENT CREATE $client READABLE ON_CLIENT_DATA
END

# Handle client data
PROC ON_CLIENT_DATA WITH client_fd DO
    GLOBAL clients
    SET data [SOCKET RECV $client_fd 1024]
    
    IF [EXPR [STRING LENGTH $data] == 0] THEN
        # Client disconnected
        PUTS "Client $client_fd disconnected"
        SOCKET CLOSE $client_fd
        EVENT DELETE $client_fd
        SET clients [LREMOVE $clients $client_fd]
    ELSE
        # Echo data back
        SOCKET SEND $client_fd $data
    END
END

# Create server
SET server [SOCKET SERVER 9999]
EVENT CREATE $server READABLE ON_ACCEPT

PUTS "Echo server listening on port 9999"
EVENT LOOP
\end{lstlisting}

\subsection{Repeating Timer}

\begin{lstlisting}[style=bcl]
GLOBAL tick_count
SET tick_count 0

PROC ON_TICK DO
    GLOBAL tick_count
    INCR tick_count
    
    PUTS "Tick $tick_count"
    
    # Re-register for next tick
    IF [EXPR $tick_count < 10] THEN
        EVENT TIMER 1000 ON_TICK
    ELSE
        PUTS "Done!"
        EVENT STOP
    END
END

# Start timer
EVENT TIMER 1000 ON_TICK
EVENT LOOP
\end{lstlisting}

\subsection{File Monitoring}

\begin{lstlisting}[style=bcl]
# Open file for reading
SET fd [OPEN "logfile.txt" r]

PROC ON_LOG_DATA WITH log_fd DO
    SET line [GETS $log_fd]
    
    IF [EOF $log_fd] THEN
        PUTS "End of file"
        CLOSE $log_fd
        EVENT DELETE $log_fd
        EVENT STOP
    ELSE
        PUTS "Log: $line"
    END
END

EVENT CREATE $fd READABLE ON_LOG_DATA
EVENT LOOP
\end{lstlisting}

\section{Integration with Extensions}

The event system is designed to work with dynamic extensions. Extensions can:

\begin{enumerate}
    \item Return file descriptors that can be monitored with \texttt{EVENT CREATE}
    \item Register custom event backends (epoll, kqueue, etc.)
    \item Provide GPIO, signal, or other event sources
\end{enumerate}

\textbf{Example - SOCKET Extension:}

The SOCKET extension returns file descriptors that integrate seamlessly:

\begin{lstlisting}[style=bcl]
LOAD "extensions/socket.so"

SET server [SOCKET SERVER 8080]  # Returns FD
EVENT CREATE $server READABLE ON_ACCEPT

SET client [SOCKET ACCEPT $server]  # Returns FD
EVENT CREATE $client READABLE ON_DATA
\end{lstlisting}

\section{Best Practices}

\subsection{Callback Parameters}

I/O event callbacks receive the file descriptor as a parameter:

\begin{lstlisting}[style=bcl]
# Correct - uses parameter
PROC ON_READ WITH fd DO
    SET data [READ $fd 1024]
END

# Incorrect - relies on global variable
PROC ON_READ DO
    GLOBAL socket
    SET data [READ $socket 1024]
END
\end{lstlisting}

\subsection{Error Handling}

Always check for errors and cleanup:

\begin{lstlisting}[style=bcl]
PROC ON_DATA WITH fd DO
    SET data [SOCKET RECV $fd 1024]
    
    IF [EXPR [STRING LENGTH $data] == 0] THEN
        # Connection closed or error
        SOCKET CLOSE $fd
        EVENT DELETE $fd
    ELSE
        # Process data
    END
END
\end{lstlisting}

\subsection{Global Variables in Callbacks}

Use explicit GLOBAL declarations for shared state:

\begin{lstlisting}[style=bcl]
GLOBAL connection_count

PROC ON_ACCEPT WITH server_fd DO
    GLOBAL connection_count
    INCR connection_count
    # ...
END
\end{lstlisting}

\section{Performance Considerations}

\begin{itemize}
    \item \textbf{Timer Precision}: Millisecond accuracy with ±10\% variance acceptable
    \item \textbf{select() Limitations}: Maximum ~1000 file descriptors on most systems
    \item \textbf{Extension Backends}: Use epoll (Linux) or kqueue (BSD/macOS) extensions for >1000 connections
    \item \textbf{Callback Overhead}: Keep callbacks fast; long-running operations should use timers
\end{itemize}

\section{Troubleshooting}

\subsection{Events Not Firing}

\begin{enumerate}
    \item Verify event is registered with \texttt{EVENT INFO}
    \item Check file descriptor is valid and open
    \item Ensure \texttt{EVENT PROCESS} or \texttt{EVENT LOOP} is called
    \item Verify callback procedure name is correct
\end{enumerate}

\subsection{Memory Leaks}

Always cleanup when done:

\begin{lstlisting}[style=bcl]
# Always delete events before closing FD
EVENT DELETE $fd
CLOSE $fd  # or SOCKET CLOSE
\end{lstlisting}

\subsection{Callback Errors}

If a callback throws an error, the event system continues processing. Check logs for error messages.

\section{See Also}

\begin{itemize}
    \item Chapter~\ref{ch:system}: System Integration Commands
    \item \texttt{docs/EVENT\_SYSTEM.md}: Complete event system documentation
    \item \texttt{docs/extensions/main.pdf}: Extension development guide
    \item \texttt{extensions/README.md}: SOCKET extension documentation
\end{itemize}
