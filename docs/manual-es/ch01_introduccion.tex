% ============================================================================
% CAPÍTULO 1: INTRODUCCIÓN
% ============================================================================

\chapter{Introducción}
\label{ch:introduction}

\section{¿Qué es BCL?}

BCL (Basic Command Language) es un lenguaje de scripting interpretado y ligero diseñado para combinar el poder de Tcl con la legibilidad de BASIC. Fue creado con los siguientes objetivos en mente:

\begin{itemize}
  \item \textbf{Simplicidad}: Fácil de aprender para principiantes
  \item \textbf{Portabilidad}: Se ejecuta tanto en PC como en sistemas embebidos
  \item \textbf{Flexibilidad}: Todo es una cadena, lo que hace que la manipulación de datos sea directa
  \item \textbf{Expresividad}: Estructura de comandos inspirada en Tcl con palabras clave al estilo BASIC
  \item \textbf{Insensibilidad a mayúsculas/minúsculas}: Los comandos y palabras clave no distinguen entre mayúsculas y minúsculas
\end{itemize}

\begin{notebox}
BCL trata \textbf{todo como una cadena}. Números, listas e incluso código están representados como texto. Esto simplifica el modelo del lenguaje y facilita el trabajo con datos de diversas fuentes.
\end{notebox}

\section{¿Por qué BCL?}

BCL llena un nicho único en el ecosistema de lenguajes de scripting:

\begin{itemize}
  \item \textbf{Para principiantes}: La sintaxis al estilo BASIC (\cmd{IF...THEN...END}, \cmd{WHILE...DO...END}) es más intuitiva que las llaves
  \item \textbf{Para usuarios de Tcl}: Estructura de comandos y modelo de evaluación familiar
  \item \textbf{Para sistemas embebidos}: Intérprete compacto adecuado para entornos con recursos limitados
  \item \textbf{Para automatización}: E/S de archivos simple, procesamiento de cadenas e interacción con el sistema
\end{itemize}

\section{Instalación y Primeros Pasos}

\subsection{Instalación}

BCL puede instalarse en varias plataformas. La distribución estándar incluye:

\begin{itemize}
  \item El ejecutable del intérprete BCL (\file{bcl})
  \item Scripts de la biblioteca estándar
  \item Programas de ejemplo
  \item Documentación
\end{itemize}

Consulte el Apéndice~\ref{app:installation} para obtener instrucciones detalladas de instalación para su plataforma.

\subsection{Ejecución de Programas BCL}

Los programas BCL pueden ejecutarse de dos formas:

\begin{enumerate}
  \item \textbf{Modo script}: Ejecutar un archivo de script BCL
  \begin{lstlisting}[language=bash]
bcl myscript.bcl
  \end{lstlisting}

  \item \textbf{Modo interactivo (REPL)}: Iniciar el intérprete BCL sin argumentos
  \begin{lstlisting}[language=bash]
bcl
  \end{lstlisting}
\end{enumerate}

\subsection{Tu Primer Programa BCL}

Comencemos con el tradicional programa "¡Hola, Mundo!":

\begin{examplebox}[title=Hola Mundo]
\begin{lstlisting}[language=BCL]
# This is a comment - my first BCL program
PUTS "Hello, World!"
\end{lstlisting}

\textbf{Salida:}
\begin{verbatim}
Hello, World!
\end{verbatim}

\textbf{Explicación:} El comando \cmd{PUTS} imprime texto en la consola seguido de una nueva línea. El texto encerrado entre comillas dobles se trata como una cadena literal.
\end{examplebox}

\subsection{Usando el REPL}

El REPL (Read-Eval-Print Loop) es el modo interactivo de BCL. Es perfecto para experimentar con comandos y probar pequeños fragmentos de código.

\begin{examplebox}[title=Sesión REPL]
\begin{verbatim}
$ bcl
BCL> PUTS "Hello from REPL"
Hello from REPL
BCL> SET x 42
BCL> PUTS "The answer is $x"
The answer is 42
BCL> EXIT
\end{verbatim}

\textbf{Explicación:} En modo REPL, escribes comandos y ves resultados inmediatos. El comando \cmd{EXIT} cierra el intérprete.
\end{examplebox}

\begin{tipbox}
¡Usa el REPL para probar comandos BCL mientras lees este manual. Es una excelente manera de aprender interactivamente!
\end{tipbox}

\section{Filosofía de BCL}

Comprender la filosofía central de BCL te ayudará a escribir mejor código:

\begin{enumerate}
  \item \textbf{Todo es una cadena}: Todos los valores son texto. Las operaciones determinan cómo se interpretan las cadenas (como números, listas, etc.)

  \item \textbf{Los comandos devuelven valores}: Casi todos los comandos BCL devuelven un valor que puede ser usado por otros comandos

  \item \textbf{Sustitución antes de evaluación}: Los valores de variables (usando \$) y los resultados de comandos (usando []) se sustituyen antes de que el comando se ejecute

  \item \textbf{Los bloques terminan con END}: Las estructuras de control usan palabras clave \cmd{END} explícitas en lugar de llaves o indentación

  \item \textbf{Insensible a mayúsculas/minúsculas}: \cmd{PUTS}, \cmd{puts}, y \cmd{Puts} son todos el mismo comando
\end{enumerate}
