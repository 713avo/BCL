% ============================================================================
% CAPÍTULO 3: VARIABLES Y DATOS
% ============================================================================

\chapter{Variables y Datos}
\label{ch:variables}

Este capítulo explora en profundidad el sistema de variables de BCL, incluyendo creación, modificación, alcance y técnicas avanzadas.

\section{Crear y Modificar Variables}

\subsection{SET: La Base}

El comando \cmd{SET} se usa tanto para crear como para modificar variables.

\textbf{Sintaxis:}
\begin{lstlisting}[language=BCL]
SET variablename value        # Assign value
SET variablename              # Return current value
\end{lstlisting}

\begin{examplebox}[title=Ejemplos de SET]
\begin{lstlisting}[language=BCL]
# Creating variables
SET counter 0
SET message "Hello"
SET pi 3.14159

# Modifying variables
SET counter 10
SET message "Goodbye"

# Reading values
PUTS [SET counter]  # Prints: 10
\end{lstlisting}
\end{examplebox}

\subsection{UNSET: Eliminar Variables}

El comando \cmd{UNSET} elimina una variable de la memoria.

\begin{examplebox}[title=Ejemplo de UNSET]
\begin{lstlisting}[language=BCL]
SET temp "temporary data"
PUTS $temp          # Prints: temporary data

UNSET temp
# PUTS $temp        # ERROR: variable doesn't exist
\end{lstlisting}
\end{examplebox}

\begin{warningbox}
Acceder a una variable eliminada causa un error. Usa \cmd{INFO EXISTS} para verificar si una variable existe antes de usarla.
\end{warningbox}

\section{Expansión de Variables}

La expansión de variables significa reemplazar \var{varname} con el valor de la variable.

\subsection{Expansión Básica}

\begin{examplebox}[title=Expansión de Variables]
\begin{lstlisting}[language=BCL]
SET fruit "apple"
SET count 5

PUTS "I have $count ${fruit}s"  # Note: BCL uses only $var
PUTS "I have $count apples"
\end{lstlisting}

\textbf{Salida:}
\begin{verbatim}
I have 5 apples
I have 5 apples
\end{verbatim}
\end{examplebox}

\begin{notebox}
BCL usa solo la forma \texttt{\$var} para la expansión de variables. A diferencia de Tcl, \texttt{\$\{var\}} no está soportado.
\end{notebox}

\subsection{Concatenación de Cadenas}

Puedes concatenar cadenas colocando expansiones de variables una al lado de la otra:

\begin{examplebox}[title=Concatenación]
\begin{lstlisting}[language=BCL]
SET first "John"
SET last "Doe"
SET fullname $first" "$last
PUTS $fullname  # Prints: John Doe

# Alternative with variables only
SET a "Hello"
SET b "World"
SET c $a$b
PUTS $c  # Prints: HelloWorld
\end{lstlisting}
\end{examplebox}

\section{INCR: Incrementar Números}

\cmd{INCR} es un comando especializado para incrementar (o decrementar) variables numéricas.

\textbf{Sintaxis:}
\begin{lstlisting}[language=BCL]
INCR varname          # Increment by 1
INCR varname amount   # Increment by amount
\end{lstlisting}

\begin{examplebox}[title=Ejemplos de INCR]
\begin{lstlisting}[language=BCL]
SET counter 10

INCR counter       # counter is now 11
INCR counter 5     # counter is now 16
INCR counter -3    # counter is now 13 (decrement)

PUTS "Counter: $counter"
\end{lstlisting}

\textbf{Salida:}
\begin{verbatim}
Counter: 13
\end{verbatim}
\end{examplebox}

\section{APPEND: Construir Cadenas}

\cmd{APPEND} agrega texto al final de una variable, modificándola en su lugar.

\textbf{Sintaxis:}
\begin{lstlisting}[language=BCL]
APPEND varname value1 value2 ...
\end{lstlisting}

\begin{examplebox}[title=Ejemplos de APPEND]
\begin{lstlisting}[language=BCL]
SET message "Hello"
APPEND message " " "World" "!"
PUTS $message  # Prints: Hello World!

# Building strings in a loop
SET result ""
SET i 1
WHILE $i <= 5 DO
  APPEND result $i " "
  INCR i
END
PUTS $result  # Prints: 1 2 3 4 5
\end{lstlisting}
\end{examplebox}

\begin{tipbox}
\cmd{APPEND} es más eficiente que operaciones \cmd{SET} repetidas cuando se construyen cadenas grandes en bucles.
\end{tipbox}

\section{Alcance de Variables}

\subsection{Variables Globales}

Las variables creadas fuera de procedimientos son globales—pueden accederse desde cualquier parte de tu programa.

\begin{lstlisting}[language=BCL]
SET global_var "I am global"

PROC test DO
  # Cannot access global_var here without GLOBAL declaration
END
\end{lstlisting}

\subsection{Variables Locales}

Las variables creadas con \cmd{SET} dentro de un procedimiento son locales—solo existen dentro de ese procedimiento.

\begin{examplebox}[title=Local vs Global]
\begin{lstlisting}[language=BCL]
SET outside "global"

PROC demo DO
  SET inside "local"
  PUTS "Inside proc: $inside"
END

demo
PUTS "Outside proc: $outside"
# PUTS $inside  # ERROR: inside doesn't exist here
\end{lstlisting}

\textbf{Salida:}
\begin{verbatim}
Inside proc: local
Outside proc: global
\end{verbatim}
\end{examplebox}

\subsection{El Comando GLOBAL}

Para acceder o modificar variables globales desde dentro de un procedimiento, usa \cmd{GLOBAL}.

\begin{examplebox}[title=Usando GLOBAL]
\begin{lstlisting}[language=BCL]
SET score 0

PROC add_points WITH points DO
  GLOBAL score
  INCR score $points
END

add_points 10
add_points 5
PUTS "Total score: $score"  # Prints: Total score: 15
\end{lstlisting}
\end{examplebox}

\section{Ejemplos Prácticos}

\subsection{Ejemplo de Contador}

\begin{examplebox}[title=Contador de Visitas]
\begin{lstlisting}[language=BCL]
SET hits 0

PROC record_hit DO
  GLOBAL hits
  INCR hits
  PUTS "Hit number $hits recorded"
END

record_hit
record_hit
record_hit
PUTS "Total hits: $hits"
\end{lstlisting}

\textbf{Salida:}
\begin{verbatim}
Hit number 1 recorded
Hit number 2 recorded
Hit number 3 recorded
Total hits: 3
\end{verbatim}
\end{examplebox}

\subsection{Ejemplo de Constructor de Texto}

\begin{examplebox}[title=Construir un Reporte]
\begin{lstlisting}[language=BCL]
SET report ""
APPEND report "=== SYSTEM REPORT ===\n"
APPEND report "Date: " [CLOCK FORMAT [CLOCK SECONDS]] "\n"
APPEND report "Status: OK\n"
APPEND report "===================="

PUTS $report
\end{lstlisting}
\end{examplebox}
