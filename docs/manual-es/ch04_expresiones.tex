% CAPÍTULO 4: EXPRESIONES Y MATEMÁTICAS
% ============================================================================

\chapter{Expresiones y Matemáticas}
\label{ch:expressions}

BCL proporciona evaluación de expresiones matemáticas y lógicas potente a través del comando \cmd{EXPR}.

\section{El Comando EXPR}

\cmd{EXPR} evalúa expresiones matemáticas y lógicas.

\textbf{Sintaxis:}
\begin{lstlisting}[language=BCL]
EXPR expression
\end{lstlisting}

\begin{examplebox}[title=Aritmética Básica]
\begin{lstlisting}[language=BCL]
SET result [EXPR 2 + 3]
PUTS $result  # Prints: 5

SET x 10
SET y 3
SET sum [EXPR $x + $y]
SET product [EXPR $x * $y]
SET quotient [EXPR $x / $y]

PUTS "Sum: $sum"          # Prints: Sum: 13
PUTS "Product: $product"  # Prints: Product: 30
PUTS "Quotient: $quotient"  # Prints: Quotient: 3.333...
\end{lstlisting}
\end{examplebox}

\section{Operadores Aritméticos}

\begin{table}[h]
\centering
\caption{Operadores Aritméticos en BCL}
\label{tab:arithmetic_ops}
\begin{tabular}{lll}
\toprule
\textbf{Operador} & \textbf{Operación} & \textbf{Ejemplo} \\
\midrule
\texttt{+} & Suma & \texttt{EXPR 5 + 3} $\rightarrow$ 8 \\
\texttt{-} & Resta & \texttt{EXPR 5 - 3} $\rightarrow$ 2 \\
\texttt{*} & Multiplicación & \texttt{EXPR 5 * 3} $\rightarrow$ 15 \\
\texttt{/} & División & \texttt{EXPR 5 / 2} $\rightarrow$ 2.5 \\
\texttt{\%} & Módulo (resto) & \texttt{EXPR 5 \% 3} $\rightarrow$ 2 \\
\texttt{\^{}} & Potencia & \texttt{EXPR 2 \^{} 8} $\rightarrow$ 256 \\
\bottomrule
\end{tabular}
\end{table}

\begin{examplebox}[title=Usando Operadores]
\begin{lstlisting}[language=BCL]
# Complex expression with parentheses
SET result [EXPR (10 + 5) * 2 - 3]
PUTS $result  # Prints: 27

# Power operator
SET squared [EXPR 5 ^ 2]
PUTS "5 squared is $squared"  # Prints: 5 squared is 25

# Modulo for even/odd check
SET num 17
SET remainder [EXPR $num % 2]
IF $remainder == 0 THEN
  PUTS "$num is even"
ELSE
  PUTS "$num is odd"
END
\end{lstlisting}
\end{examplebox}

\section{Operadores de Comparación}

\begin{table}[h]
\centering
\caption{Operadores de Comparación}
\label{tab:comparison_ops}
\begin{tabular}{lll}
\toprule
\textbf{Operador} & \textbf{Significado} & \textbf{Ejemplo} \\
\midrule
\texttt{=} & Igual a & \texttt{\$a == 5} \\
\texttt{!=} & No igual a & \texttt{\$a != 5} \\
\texttt{<} & Menor que & \texttt{\$a < 10} \\
\texttt{<=} & Menor o igual que & \texttt{\$a <= 10} \\
\texttt{>} & Mayor que & \texttt{\$a > 5} \\
\texttt{>=} & Mayor o igual que & \texttt{\$a >= 5} \\
\bottomrule
\end{tabular}
\end{table}

\begin{examplebox}[title=Comparaciones]
\begin{lstlisting}[language=BCL]
SET age 25

IF $age >= 18 THEN
  PUTS "Adult"
ELSE
  PUTS "Minor"
END

SET score 85
IF $score >= 90 THEN
  PUTS "Grade: A"
ELSEIF $score >= 80 THEN
  PUTS "Grade: B"
ELSEIF $score >= 70 THEN
  PUTS "Grade: C"
ELSE
  PUTS "Grade: F"
END
\end{lstlisting}
\end{examplebox}

\section{Operadores Lógicos}

\begin{table}[h]
\centering
\caption{Operadores Lógicos}
\label{tab:logical_ops}
\begin{tabular}{lll}
\toprule
\textbf{Operador} & \textbf{Símbolo} & \textbf{Ejemplo} \\
\midrule
AND & \texttt{\&\&} & \texttt{\$a > 0 AND \$a < 10} \\
OR & \texttt{||} & \texttt{\$a == 5 OR \$a == 10} \\
NOT & \texttt{!} & \texttt{NOT \$flag} \\
\bottomrule
\end{tabular}
\end{table}

\begin{examplebox}[title=Operaciones Lógicas]
\begin{lstlisting}[language=BCL]
SET age 25
SET has_license 1

# AND operation
IF $age >= 18 AND $has_license THEN
  PUTS "Can drive"
END

# OR operation
SET day "Saturday"
IF $day == "Saturday" OR $day == "Sunday" THEN
  PUTS "It's the weekend!"
END

# NOT operation
SET raining 0
IF NOT $raining THEN
  PUTS "No umbrella needed"
END
\end{lstlisting}
\end{examplebox}

\section{Funciones Matemáticas}

BCL incluye muchas funciones matemáticas:

\begin{table}[h]
\centering
\caption{Funciones Matemáticas}
\label{tab:math_functions}
\begin{tabular}{ll}
\toprule
\textbf{Función} & \textbf{Descripción} \\
\midrule
\texttt{abs(x)} & Valor absoluto \\
\texttt{sqrt(x)} & Raíz cuadrada \\
\texttt{pow(x,y)} & x elevado a la potencia y \\
\texttt{exp(x)} & e elevado a la potencia x \\
\texttt{log(x)} & Logaritmo base 10 \\
\texttt{ln(x)} & Logaritmo natural (base e) \\
\texttt{sin(x)} & Seno (radianes) \\
\texttt{cos(x)} & Coseno (radianes) \\
\texttt{tan(x)} & Tangente (radianes) \\
\texttt{asin(x)} & Arcoseno \\
\texttt{acos(x)} & Arcocoseno \\
\texttt{atan(x)} & Arcotangente \\
\texttt{hypo(x,y)} & Hipotenusa: sqrt(x$^2$ + y$^2$) \\
\texttt{ceil(x)} & Redondear hacia arriba \\
\texttt{floor(x)} & Redondear hacia abajo \\
\texttt{round(x)} & Redondear al entero más cercano \\
\texttt{int(x)} & Convertir a entero \\
\texttt{double(x)} & Convertir a punto flotante \\
\texttt{rand()} & Número aleatorio 0.0 a 1.0 \\
\texttt{srand(seed)} & Establecer semilla aleatoria \\
\bottomrule
\end{tabular}
\end{table}

\begin{examplebox}[title=Trigonometría]
\begin{lstlisting}[language=BCL]
SET pi 3.14159265359

# Sine of 90 degrees (pi/2 radians)
SET angle [EXPR $pi / 2]
SET sine [EXPR sin($angle)]
PUTS [FORMAT "sin(90\textdegree) = %.4f" $sine]

# Calculate hypotenuse
SET a 3
SET b 4
SET c [EXPR hypo($a, $b)]
PUTS [FORMAT "Hypotenuse of %d and %d = %.2f" $a $b $c]
\end{lstlisting}

\textbf{Salida:}
\begin{verbatim}
sin(90\textdegree) = 1.0000
Hypotenuse of 3 and 4 = 5.00
\end{verbatim}
\end{examplebox}

\begin{examplebox}[title=Números Aleatorios]
\begin{lstlisting}[language=BCL]
# Generate random integer from 1 to 10
SET random [EXPR int(rand() * 10) + 1]
PUTS "Random number: $random"

# Dice roll simulator
PROC roll_dice DO
  SET roll [EXPR int(rand() * 6) + 1]
  RETURN $roll
END

SET roll1 [roll_dice]
SET roll2 [roll_dice]
PUTS "You rolled: $roll1 and $roll2"
\end{lstlisting}
\end{examplebox}

\section{Ejemplos Prácticos}

\subsection{Calculadora de Círculos}

\begin{examplebox}[title=Área y Circunferencia de un Círculo]
\begin{lstlisting}[language=BCL]
PROC circle_stats WITH radius DO
  SET pi 3.14159265359
  SET area [EXPR $pi * $radius * $radius]
  SET circumference [EXPR 2 * $pi * $radius]

  PUTS [FORMAT "Radius: %.2f" $radius]
  PUTS [FORMAT "Area: %.2f" $area]
  PUTS [FORMAT "Circumference: %.2f" $circumference]
END

circle_stats 5.0
\end{lstlisting}

\textbf{Salida:}
\begin{verbatim}
Radius: 5.00
Area: 78.54
Circumference: 31.42
\end{verbatim}
\end{examplebox}

\subsection{Resolución de Ecuaciones Cuadráticas}

\begin{examplebox}[title=Resolviendo ax$^2$ + bx + c = 0]
\begin{lstlisting}[language=BCL]
PROC solve_quadratic WITH a b c DO
  # Calculate discriminant
  SET disc [EXPR $b*$b - 4*$a*$c]

  IF $disc < 0 THEN
    PUTS "No real solutions"
    RETURN
  END

  # Calculate solutions
  SET x1 [EXPR (-$b + sqrt($disc)) / (2*$a)]
  SET x2 [EXPR (-$b - sqrt($disc)) / (2*$a)]

  PUTS [FORMAT "x1 = %.4f" $x1]
  PUTS [FORMAT "x2 = %.4f" $x2]
END

# Solve x$^2$ - 5x + 6 = 0
solve_quadratic 1 -5 6
\end{lstlisting}

\textbf{Salida:}
\begin{verbatim}
x1 = 3.0000
x2 = 2.0000
\end{verbatim}
\end{examplebox}
