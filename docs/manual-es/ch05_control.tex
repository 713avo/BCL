% ============================================================================
% CAPÍTULO 5: ESTRUCTURAS DE CONTROL
% ============================================================================

\chapter{Estructuras de Control}
\label{ch:control}

Las estructuras de control te permiten cambiar el orden en que tu programa ejecuta comandos basándose en condiciones o repetición.

\section{IF...THEN...ELSE...END}

La instrucción \cmd{IF} ejecuta código condicionalmente.

\textbf{Sintaxis:}
\begin{lstlisting}[language=BCL]
IF condition THEN
  commands
[ELSEIF condition THEN
  commands]
[ELSE
  commands]
END
\end{lstlisting}

\begin{examplebox}[title=IF Simple]
\begin{lstlisting}[language=BCL]
SET temperature 25

IF $temperature > 30 THEN
  PUTS "It's hot!"
ELSEIF $temperature > 20 THEN
  PUTS "It's warm"
ELSEIF $temperature > 10 THEN
  PUTS "It's cool"
ELSE
  PUTS "It's cold!"
END
\end{lstlisting}

\textbf{Salida:}
\begin{verbatim}
It's warm
\end{verbatim}
\end{examplebox}

\begin{examplebox}[title=IF Anidado]
\begin{lstlisting}[language=BCL]
SET age 25
SET student 1

IF $age < 18 THEN
  PUTS "Minor - discounted ticket"
ELSE
  IF $student THEN
    PUTS "Adult student - discounted ticket"
  ELSE
    PUTS "Regular adult ticket"
  END
END
\end{lstlisting}
\end{examplebox}

\section{Bucles WHILE}

\cmd{WHILE} repite código mientras una condición sea verdadera.

\textbf{Sintaxis:}
\begin{lstlisting}[language=BCL]
WHILE condition DO
  commands
END
\end{lstlisting}

\begin{examplebox}[title=Cuenta Regresiva]
\begin{lstlisting}[language=BCL]
SET count 5

WHILE $count > 0 DO
  PUTS "Countdown: $count"
  INCR count -1
END

PUTS "Liftoff!"
\end{lstlisting}

\textbf{Salida:}
\begin{verbatim}
Countdown: 5
Countdown: 4
Countdown: 3
Countdown: 2
Countdown: 1
Liftoff!
\end{verbatim}
\end{examplebox}

\begin{examplebox}[title=Validación de Entrada]
\begin{lstlisting}[language=BCL]
SET valid 0

WHILE NOT $valid DO
  PUTS "Enter a number between 1 and 10:"
  SET input [GETS stdin]

  IF $input >= 1 AND $input <= 10 THEN
    SET valid 1
    PUTS "Thank you! You entered $input"
  ELSE
    PUTS "Invalid input. Try again."
  END
END
\end{lstlisting}
\end{examplebox}

\section{Bucles FOR}

BCL soporta dos estilos de bucles \cmd{FOR}.

\subsection{FOR con Variable Explícita}

\begin{examplebox}[title=Bucle FOR - Variable Explícita]
\begin{lstlisting}[language=BCL]
FOR [SET i 1] TO $i <= 10 DO
  PUTS "Iteration $i"
  INCR i
END
\end{lstlisting}
\end{examplebox}

\subsection{FOR con Contador Interno}

\begin{examplebox}[title=Bucle FOR - Contador Interno]
\begin{lstlisting}[language=BCL]
FOR 1 TO 10 DO
  PUTS "Number: $__FOR"
END

# With STEP
FOR 0 TO 100 STEP 10 DO
  PUTS "Value: $__FOR"
END
\end{lstlisting}

\textbf{Salida:}
\begin{verbatim}
Number: 1
Number: 2
...
Number: 10
Value: 0
Value: 10
...
Value: 100
\end{verbatim}
\end{examplebox}

\begin{notebox}
La variable de contador interno se llama \var{\_\_FOR} y es creada automáticamente por BCL.
\end{notebox}

\section{Bucles FOREACH}

\cmd{FOREACH} itera sobre elementos de lista.

\textbf{Sintaxis:}
\begin{lstlisting}[language=BCL]
FOREACH variable IN list DO
  commands
END
\end{lstlisting}

\begin{examplebox}[title=Iterar Listas]
\begin{lstlisting}[language=BCL]
SET colors [LIST red green blue yellow]

FOREACH color IN $colors DO
  PUTS "Color: $color"
END
\end{lstlisting}

\textbf{Salida:}
\begin{verbatim}
Color: red
Color: green
Color: blue
Color: yellow
\end{verbatim}
\end{examplebox}

\begin{examplebox}[title=Procesar Datos]
\begin{lstlisting}[language=BCL]
SET scores [LIST 85 92 78 95 88]
SET total 0
SET count 0

FOREACH score IN $scores DO
  SET total [EXPR $total + $score]
  INCR count
END

SET average [EXPR $total / $count]
PUTS [FORMAT "Average score: %.2f" $average]
\end{lstlisting}

\textbf{Salida:}
\begin{verbatim}
Average score: 87.60
\end{verbatim}
\end{examplebox}

\section{Instrucción SWITCH}

\cmd{SWITCH} selecciona uno de muchos bloques de código para ejecutar.

\textbf{Sintaxis:}
\begin{lstlisting}[language=BCL]
SWITCH expression DO
  CASE value1
    commands
  CASE value2
    commands
  DEFAULT
    commands
END
\end{lstlisting}

\begin{examplebox}[title=Ejemplo de SWITCH]
\begin{lstlisting}[language=BCL]
SET day "Tuesday"

SWITCH $day DO
  CASE "Monday"
    PUTS "Start of work week"
  CASE "Tuesday"
    PUTS "Second day"
  CASE "Wednesday"
    PUTS "Midweek"
  CASE "Thursday"
    PUTS "Almost Friday"
  CASE "Friday"
    PUTS "Last work day!"
  DEFAULT
    PUTS "Weekend!"
END
\end{lstlisting}

\textbf{Salida:}
\begin{verbatim}
Second day
\end{verbatim}
\end{examplebox}

\begin{examplebox}[title=Sistema de Menú]
\begin{lstlisting}[language=BCL]
PUTS "Select an option:"
PUTS "1. New file"
PUTS "2. Open file"
PUTS "3. Save file"
PUTS "4. Exit"

SET choice [GETS stdin]

SWITCH $choice DO
  CASE "1"
    PUTS "Creating new file..."
  CASE "2"
    PUTS "Opening file..."
  CASE "3"
    PUTS "Saving file..."
  CASE "4"
    PUTS "Goodbye!"
    EXIT
  DEFAULT
    PUTS "Invalid choice"
END
\end{lstlisting}
\end{examplebox}

\section{BREAK y CONTINUE}

\subsection{BREAK: Salir del Bucle Anticipadamente}

\cmd{BREAK} termina el bucle más interno inmediatamente.

\begin{examplebox}[title=Usando BREAK]
\begin{lstlisting}[language=BCL]
# Search for a number
SET target 7
SET found 0

FOR 1 TO 10 DO
  IF $__FOR == $target THEN
    SET found $__FOR
    BREAK
  END
END

IF $found THEN
  PUTS "Found $target"
ELSE
  PUTS "Not found"
END
\end{lstlisting}
\end{examplebox}

\subsection{CONTINUE: Saltar a la Siguiente Iteración}

\cmd{CONTINUE} salta el resto de la iteración actual y va a la siguiente.

\begin{examplebox}[title=Usando CONTINUE]
\begin{lstlisting}[language=BCL]
# Print odd numbers only
FOR 1 TO 10 DO
  SET num $__FOR
  SET remainder [EXPR $num % 2]

  IF $remainder == 0 THEN
    CONTINUE  # Skip even numbers
  END

  PUTS $num
END
\end{lstlisting}

\textbf{Salida:}
\begin{verbatim}
1
3
5
7
9
\end{verbatim}
\end{examplebox}

\section{EXIT: Terminar el Programa}

\cmd{EXIT} termina el programa completo (o sesión REPL).

\begin{lstlisting}[language=BCL]
EXIT          # Exit with code 0
EXIT 1        # Exit with code 1 (error)
\end{lstlisting}

\section{Ejemplos Completos}

\subsection{Juego de Adivinar Números}

\begin{examplebox}[title=Juego de Adivinanzas]
\begin{lstlisting}[language=BCL]
# Generate random number 1-100
SET secret [EXPR int(rand() * 100) + 1]
SET guesses 0
SET found 0

PUTS "I'm thinking of a number between 1 and 100"

WHILE NOT $found DO
  PUTS "Enter your guess:"
  SET guess [GETS stdin]
  INCR guesses

  IF $guess == $secret THEN
    SET found 1
    PUTS "Correct! You won in $guesses guesses!"
  ELSEIF $guess < $secret THEN
    PUTS "Too low!"
  ELSE
    PUTS "Too high!"
  END
END
\end{lstlisting}
\end{examplebox}

\subsection{FizzBuzz}

\begin{examplebox}[title=FizzBuzz Clásico]
\begin{lstlisting}[language=BCL]
FOR 1 TO 100 DO
  SET num $__FOR
  SET by3 [EXPR $num % 3]
  SET by5 [EXPR $num % 5]

  IF $by3 == 0 AND $by5 == 0 THEN
    PUTS "FizzBuzz"
  ELSEIF $by3 == 0 THEN
    PUTS "Fizz"
  ELSEIF $by5 == 0 THEN
    PUTS "Buzz"
  ELSE
    PUTS $num
  END
END
\end{lstlisting}
\end{examplebox}
