% ============================================================================
% CAPÍTULO 6: PROCEDIMIENTOS (FUNCIONES)
% ============================================================================

\chapter{Procedimientos (Funciones)}
\label{ch:procedures}

Los procedimientos (también llamados funciones en otros lenguajes) te permiten empaquetar código en bloques reutilizables.

\section{Definir Procedimientos}

\textbf{Sintaxis con parámetros:}
\begin{lstlisting}[language=BCL]
PROC name WITH param1 param2 ... DO
  commands
  RETURN value
END
\end{lstlisting}

\textbf{Sintaxis sin parámetros:}
\begin{lstlisting}[language=BCL]
PROC name DO
  commands
  RETURN value
END
\end{lstlisting}

\begin{notebox}
Cuando un procedimiento no tiene parámetros, la palabra clave \cmd{WITH} puede omitirse.
\end{notebox}

\begin{examplebox}[title=Procedimiento con Parámetro]
\begin{lstlisting}[language=BCL]
PROC greet WITH name DO
  PUTS "Hello, $name!"
END

greet "Alice"
greet "Bob"
\end{lstlisting}

\textbf{Salida:}
\begin{verbatim}
Hello, Alice!
Hello, Bob!
\end{verbatim}
\end{examplebox}

\begin{examplebox}[title=Procedimiento sin Parámetros]
\begin{lstlisting}[language=BCL]
PROC show_version DO
  PUTS "BCL Interpreter v1.5.0"
END

show_version
\end{lstlisting}

\textbf{Salida:}
\begin{verbatim}
BCL Interpreter v1.5.0
\end{verbatim}
\end{examplebox}

\begin{notebox}
Los procedimientos se invocan por su nombre solo—no se necesita palabra clave \cmd{CALL}.
\end{notebox}

\section{Parámetros}

\subsection{Parámetros Fijos}

Los parámetros se declaran en la cláusula \cmd{WITH} y son accesibles como variables dentro del procedimiento.

\begin{examplebox}[title=Múltiples Parámetros]
\begin{lstlisting}[language=BCL]
PROC add WITH a b DO
  SET result [EXPR $a + $b]
  RETURN $result
END

SET sum [add 5 3]
PUTS "5 + 3 = $sum"
\end{lstlisting}

\textbf{Salida:}
\begin{verbatim}
5 + 3 = 8
\end{verbatim}
\end{examplebox}

\subsection{Parámetros Opcionales}

Los parámetros opcionales se prefijan con \texttt{@}. Pueden o no proporcionarse al llamar al procedimiento.

\begin{examplebox}[title=Parámetros Opcionales]
\begin{lstlisting}[language=BCL]
PROC greet WITH name @title DO
  IF [INFO EXISTS title] THEN
    PUTS "Hello, $title $name"
  ELSE
    PUTS "Hello, $name"
  END
END

greet "Smith"
greet "Smith" "Dr."
\end{lstlisting}

\textbf{Salida:}
\begin{verbatim}
Hello, Smith
Hello, Dr. Smith
\end{verbatim}
\end{examplebox}

\begin{tipbox}
Usa \cmd{INFO EXISTS} para verificar si se proporcionó un parámetro opcional.
\end{tipbox}

\section{Valores de Retorno}

El comando \cmd{RETURN} sale de un procedimiento y opcionalmente devuelve un valor.

\begin{examplebox}[title=Devolver Valores]
\begin{lstlisting}[language=BCL]
PROC square WITH n DO
  SET result [EXPR $n * $n]
  RETURN $result
END

PROC is_even WITH n DO
  SET remainder [EXPR $n % 2]
  IF $remainder == 0 THEN
    RETURN 1  # true
  ELSE
    RETURN 0  # false
  END
END

SET s [square 7]
PUTS "7 squared = $s"

IF [is_even 10] THEN
  PUTS "10 is even"
END
\end{lstlisting}

\textbf{Salida:}
\begin{verbatim}
7 squared = 49
10 is even
\end{verbatim}
\end{examplebox}

\begin{notebox}
Si \cmd{RETURN} se llama sin un valor, o si un procedimiento alcanza su \cmd{END} sin devolver nada, devuelve una cadena vacía.
\end{notebox}

\section{Alcance de Variables en Procedimientos}

\subsection{Variables Locales}

Las variables creadas con \cmd{SET} dentro de un procedimiento son locales—solo existen dentro de ese procedimiento.

\begin{examplebox}[title=Variables Locales]
\begin{lstlisting}[language=BCL]
PROC calculate WITH x DO
  SET double [EXPR $x * 2]  # local variable
  SET triple [EXPR $x * 3]  # local variable
  PUTS "Inside: double=$double, triple=$triple"
END

calculate 5
# PUTS $double  # ERROR: double doesn't exist here
\end{lstlisting}
\end{examplebox}

\subsection{Acceder a Variables Globales}

Usa \cmd{GLOBAL} para acceder o modificar variables globales desde dentro de un procedimiento.

\begin{examplebox}[title=Variables Globales en Procedimientos]
\begin{lstlisting}[language=BCL]
SET counter 0

PROC increment WITH amount DO
  GLOBAL counter
  INCR counter $amount
  PUTS "Counter is now: $counter"
END

increment 5
increment 3
PUTS "Final counter: $counter"
\end{lstlisting}

\textbf{Salida:}
\begin{verbatim}
Counter is now: 5
Counter is now: 8
Final counter: 8
\end{verbatim}
\end{examplebox}

\section{Procedimientos Recursivos}

Los procedimientos pueden llamarse a sí mismos—esto se llama recursión.

\begin{examplebox}[title=Factorial (Recursivo)]
\begin{lstlisting}[language=BCL]
PROC factorial WITH n DO
  IF $n <= 1 THEN
    RETURN 1
  ELSE
    SET prev [factorial [EXPR $n - 1]]
    RETURN [EXPR $n * $prev]
  END
END

PUTS "5! = [factorial 5]"
PUTS "10! = [factorial 10]"
\end{lstlisting}

\textbf{Salida:}
\begin{verbatim}
5! = 120
10! = 3628800
\end{verbatim}
\end{examplebox}

\begin{warningbox}
¡Los procedimientos recursivos deben tener un caso base (condición de terminación) para evitar recursión infinita!
\end{warningbox}

\begin{examplebox}[title=Fibonacci (Recursivo)]
\begin{lstlisting}[language=BCL]
PROC fib WITH n DO
  IF $n <= 1 THEN
    RETURN $n
  ELSE
    SET n1 [EXPR $n - 1]
    SET n2 [EXPR $n - 2]
    SET f1 [fib $n1]
    SET f2 [fib $n2]
    RETURN [EXPR $f1 + $f2]
  END
END

# Print first 10 Fibonacci numbers
FOR 0 TO 9 DO
  SET f [fib $__FOR]
  PUTS "fib($__FOR) = $f"
END
\end{lstlisting}
\end{examplebox}

\section{Ejemplos Prácticos}

\subsection{Convertidor de Temperatura}

\begin{examplebox}[title=Celsius a Fahrenheit]
\begin{lstlisting}[language=BCL]
PROC celsius_to_fahrenheit WITH celsius DO
  SET fahrenheit [EXPR ($celsius * 9.0 / 5.0) + 32]
  RETURN $fahrenheit
END

PROC fahrenheit_to_celsius WITH fahrenheit DO
  SET celsius [EXPR ($fahrenheit - 32) * 5.0 / 9.0]
  RETURN $celsius
END

SET c 25
SET f [celsius_to_fahrenheit $c]
PUTS [FORMAT "%d\textdegreeC = %.1f\textdegreeF" $c $f]

SET f2 100
SET c2 [fahrenheit_to_celsius $f2]
PUTS [FORMAT "%d\textdegreeF = %.1f\textdegreeC" $f2 $c2]
\end{lstlisting}

\textbf{Salida:}
\begin{verbatim}
25\textdegreeC = 77.0\textdegreeF
100\textdegreeF = 37.8\textdegreeC
\end{verbatim}
\end{examplebox}

\subsection{Utilidades de Cadenas}

\begin{examplebox}[title=Procedimientos Auxiliares de Cadenas]
\begin{lstlisting}[language=BCL]
PROC reverse_string WITH str DO
  RETURN [STRING REVERSE $str]
END

PROC capitalize WITH str DO
  SET first [STRING INDEX $str 0]
  SET rest [STRING RANGE $str 1 end]
  SET first_upper [STRING TOUPPER $first]
  SET rest_lower [STRING TOLOWER $rest]
  RETURN [STRING CAT $first_upper $rest_lower]
END

SET text "hello world"
PUTS "Original: $text"
PUTS "Reversed: [reverse_string $text]"
PUTS "Capitalized: [capitalize $text]"
\end{lstlisting}

\textbf{Salida:}
\begin{verbatim}
Original: hello world
Reversed: dlrow olleh
Capitalized: Hello world
\end{verbatim}
\end{examplebox}
