% ============================================================================
% CAPÍTULO 9: ARCHIVOS
% ============================================================================

\chapter{Archivos}
\label{ch:files}

BCL proporciona capacidades completas de E/S de archivos.

\section{Abrir y Cerrar Archivos}

\begin{table}[h]
\centering
\caption{Modos de Apertura de Archivos}
\label{tab:file_modes}
\begin{tabular}{ll}
\toprule
\textbf{Modo} & \textbf{Descripción} \\
\midrule
\texttt{R} & Lectura (el archivo debe existir) \\
\texttt{W} & Escritura (crea o trunca) \\
\texttt{A} & Agregar (crea si es necesario) \\
\texttt{RW} & Lectura y escritura \\
\bottomrule
\end{tabular}
\end{table}

\begin{examplebox}[title=E/S Básica de Archivos]
\begin{lstlisting}[language=BCL]
# Write to file
SET fh [OPEN "output.txt" W]
PUTS $fh "Line 1"
PUTS $fh "Line 2"
CLOSE $fh

# Read from file
SET fh [OPEN "output.txt" R]
SET content [READ $fh]
CLOSE $fh

PUTS "File content:"
PUTS $content
\end{lstlisting}
\end{examplebox}

\section{Leer Archivos Línea por Línea}

\begin{examplebox}[title=Lectura Línea por Línea]
\begin{lstlisting}[language=BCL]
SET fh [OPEN "data.txt" R]

SET linenum 0
WHILE [EOF $fh] = 0 DO
  SET line [GETS $fh]
  INCR linenum

  IF [STRING LENGTH $line] > 0 THEN
    PUTS "Line $linenum: $line"
  END
END

CLOSE $fh
\end{lstlisting}
\end{examplebox}

\section{Comandos de Archivo}

\begin{examplebox}[title=Comandos FILE]
\begin{lstlisting}[language=BCL]
SET filename "test.txt"

# Check existence
IF [FILE EXISTS $filename] THEN
  PUTS "File exists"

  # Get size
  SET size [FILE SIZE $filename]
  PUTS "Size: $size bytes"

  # Rename
  FILE RENAME $filename "test_backup.txt"

  # Delete
  # FILE DELETE "test_backup.txt"
END

# Get current directory
SET cwd [PWD]
PUTS "Current directory: $cwd"

# Find files matching pattern
SET txtfiles [GLOB "*.txt"]
PUTS "Text files: $txtfiles"
\end{lstlisting}
\end{examplebox}

\section{Posicionamiento en Archivos con SEEK}

El comando \texttt{SEEK} permite mover el puntero del archivo a una posición específica dentro de un archivo.

\begin{syntaxbox}
SEEK handle offset whence
\end{syntaxbox}

\begin{itemize}
\item \textbf{handle} -- Manejador de archivo desde \texttt{OPEN}
\item \textbf{offset} -- Desplazamiento en bytes (puede ser negativo para \texttt{END})
\item \textbf{whence} -- Punto de referencia:
  \begin{itemize}
  \item \texttt{START} o \texttt{SET} -- Desde el principio del archivo
  \item \texttt{CUR} o \texttt{CURRENT} -- Desde la posición actual
  \item \texttt{END} -- Desde el final del archivo
  \end{itemize}
\end{itemize}

\begin{examplebox}[title=Uso de SEEK]
\begin{lstlisting}[language=BCL]
# Abrir archivo para lectura
SET fh [OPEN "data.bin" R]

# Leer primeros 10 bytes
SET header [READ $fh 10]

# Volver al principio
SEEK $fh 0 START

# Saltar 100 bytes desde la posición actual
SEEK $fh 100 CUR

# Ir a 10 bytes antes del final
SEEK $fh -10 END

CLOSE $fh
\end{lstlisting}
\end{examplebox}
