% ============================================================================
% CAPÍTULO 10: EXPRESIONES REGULARES
% ============================================================================

\chapter{Expresiones Regulares}
\label{ch:regexp}

Las expresiones regulares (a menudo llamadas "regex" o "regexp") son patrones potentes usados para buscar, coincidir y manipular texto. Piensa en ellas como una herramienta sofisticada de "buscar y reemplazar" que puede coincidir con patrones complejos en lugar de solo texto exacto.

\section{¿Qué son las Expresiones Regulares?}

Imagina que quieres encontrar todos los números de teléfono en un documento, o validar que una dirección de correo electrónico esté formateada correctamente, o reemplazar todas las fechas de un formato a otro. Las expresiones regulares hacen estas tareas fáciles.

\begin{examplebox}[title=Analogía del Mundo Real]
Las expresiones regulares son como describir algo sin conocer su forma exacta:
\begin{itemize}
  \item "Cualquier palabra que comience con 'cat'" - coincide con "cat", "cats", "category"
  \item "Una secuencia de dígitos" - coincide con "123", "4567", "999"
  \item "Texto entre comillas" - coincide con "hello", "goodbye"
\end{itemize}
\end{examplebox}

\begin{notebox}
Si eres nuevo en programación, las expresiones regulares pueden parecer crípticas al principio. ¡No te preocupes! Comenzaremos con patrones simples y construiremos hasta los más complejos.
\end{notebox}

\section{Bloques de Construcción de Patrones Básicos}

Las expresiones regulares se construyen a partir de piezas simples. Aprendámoslas una a la vez.

\subsection{Caracteres Literales}

El patrón más simple es solo texto normal—coincide exactamente con lo que escribes.

\begin{examplebox}[title=Coincidencia Literal]
\begin{lstlisting}[language=BCL]
SET text "The cat sat on the mat"

# Match the word "cat"
IF [REGEXP "cat" $text] THEN
  PUTS "Found 'cat' in the text"
END

# Match "mat"
IF [REGEXP "mat" $text] THEN
  PUTS "Found 'mat' in the text"
END

# This won't match because case matters
IF [REGEXP "Cat" $text] THEN
  PUTS "Found 'Cat'"
ELSE
  PUTS "'Cat' not found (wrong case)"
END

# Usa la opción NOCASE para coincidencia insensible a mayúsculas
IF [REGEXP "Cat" $text NOCASE] THEN
  PUTS "Found 'Cat' (case-insensitive)"
END
\end{lstlisting}

\textbf{Salida:}
\begin{verbatim}
Found 'cat' in the text
Found 'mat' in the text
'Cat' not found (wrong case)
Found 'Cat' (case-insensitive)
\end{verbatim}
\end{examplebox}

\begin{tipbox}
Usa la opción \texttt{NOCASE} para hacer que los patrones coincidan sin importar mayúsculas/minúsculas. Coloca \texttt{NOCASE} después del argumento de texto:
\begin{lstlisting}[language=BCL]
REGEXP pattern text NOCASE
\end{lstlisting}
\end{tipbox}

\subsection{Caracteres Especiales - Los Comodines}

Algunos caracteres tienen significados especiales en las expresiones regulares:

\begin{table}[h]
\centering
\caption{Caracteres Básicos de Expresiones Regulares}
\label{tab:regex_basics}
\begin{tabular}{lp{8cm}}
\toprule
\textbf{Símbolo} & \textbf{Significado} \\
\midrule
\texttt{.} & Coincide con cualquier carácter individual \\
\texttt{*} & Coincide con 0 o más del carácter anterior \\
\texttt{+} & Coincide con 1 o más del carácter anterior \\
\texttt{?} & Coincide con 0 o 1 del carácter anterior \\
\texttt{\^{}} & Coincide con el inicio de la cadena \\
\texttt{\$} & Coincide con el final de la cadena \\
\texttt{|} & Operador OR (coincide con izquierda o derecha) \\
\texttt{(...)} & Agrupa patrones juntos \\
\texttt{[...]} & Coincide con cualquier carácter entre corchetes \\
\texttt{\textbackslash} & Escapa caracteres especiales \\
\bottomrule
\end{tabular}
\end{table}

\subsection{El Punto (.) - Cualquier Carácter}

El punto coincide con cualquier carácter individual excepto nueva línea.

\begin{examplebox}[title=Usando el Punto]
\begin{lstlisting}[language=BCL]
# Match "c.t" - c, any character, then t
SET words [LIST "cat" "cot" "cut" "cart" "ct"]

FOREACH word IN $words DO
  IF [REGEXP "c.t" $word] THEN
    PUTS "$word matches c.t"
  ELSE
    PUTS "$word does NOT match c.t"
  END
END
\end{lstlisting}

\textbf{Salida:}
\begin{verbatim}
cat matches c.t
cot matches c.t
cut matches c.t
cart matches c.t
ct does NOT match c.t
\end{verbatim}

\textbf{Explicación:} El patrón \texttt{c.t} requiere exactamente un carácter entre 'c' y 't'. "cart" coincide porque contiene "car" + "t" = "cart" que tiene el patrón.
\end{examplebox}

\subsection{Clases de Caracteres [...]}

Los corchetes coinciden con cualquier carácter de un conjunto.

\begin{examplebox}[title=Clases de Caracteres]
\begin{lstlisting}[language=BCL]
# Match c[aou]t - cat, cot, or cut
SET words [LIST "cat" "cot" "cut" "cit" "cet"]

FOREACH word IN $words DO
  IF [REGEXP "c\[aou\]t" $word] THEN
    PUTS "$word matches"
  END
END

# Match any digit [0-9]
SET text "I have 5 apples"
IF [REGEXP "\[0-9\]" $text] THEN
  PUTS "Contains a digit"
END

# Match any letter [a-z] or [A-Z]
IF [REGEXP "\[a-z\]" $text] THEN
  PUTS "Contains lowercase letters"
END
\end{lstlisting}

\textbf{Salida:}
\begin{verbatim}
cat matches
cot matches
cut matches
Contains a digit
Contains lowercase letters
\end{verbatim}
\end{examplebox}

\begin{tipbox}
Clases de caracteres comunes:
\begin{itemize}
  \item \texttt{[0-9]} - cualquier dígito
  \item \texttt{[a-z]} - cualquier letra minúscula
  \item \texttt{[A-Z]} - cualquier letra mayúscula
  \item \texttt{[a-zA-Z]} - cualquier letra
  \item \texttt{[\^{}0-9]} - cualquier cosa que NO sea un dígito
\end{itemize}
\end{tipbox}

\subsection{Repetición: *, +, ?}

Estos indican cuántas veces un patrón debe repetirse.

\begin{table}[h]
\centering
\caption{Operadores de Repetición}
\label{tab:regex_repetition}
\begin{tabular}{lp{8cm}}
\toprule
\textbf{Operador} & \textbf{Significado} \\
\midrule
\texttt{*} & 0 o más veces (opcional, puede repetir) \\
\texttt{+} & 1 o más veces (debe aparecer al menos una vez) \\
\texttt{?} & 0 o 1 vez (opcional, aparece una vez o no) \\
\texttt{\{n\}} & Exactamente n veces \\
\texttt{\{n,\}} & n o más veces \\
\texttt{\{n,m\}} & Entre n y m veces \\
\bottomrule
\end{tabular}
\end{table}

\begin{examplebox}[title=Ejemplos de Repetición]
\begin{lstlisting}[language=BCL]
# Match one or more digits
SET texts [LIST "abc" "123" "abc123" "a1b2c3"]

FOREACH text IN $texts DO
  IF [REGEXP "\[0-9\]+" $text] THEN
    PUTS "$text contains numbers"
  END
END

# Match optional minus sign followed by digits: -?[0-9]+
SET numbers [LIST "123" "-456" "78" "-9"]
FOREACH num IN $numbers DO
  IF [REGEXP "^-?\[0-9\]+$" $num] THEN
    PUTS "$num is a valid number"
  END
END

# Match "color" or "colour"
IF [REGEXP "colou?r" "color"] THEN
  PUTS "Matches 'color'"
END
IF [REGEXP "colou?r" "colour"] THEN
  PUTS "Matches 'colour'"
END
\end{lstlisting}

\textbf{Salida:}
\begin{verbatim}
123 contains numbers
abc123 contains numbers
a1b2c3 contains numbers
123 is a valid number
-456 is a valid number
78 is a valid number
-9 is a valid number
Matches 'color'
Matches 'colour'
\end{verbatim}
\end{examplebox}

\subsection{Anclas: \^{} y \$}

Las anclas coinciden con posiciones, no con caracteres.

\begin{itemize}
  \item \texttt{\^{}} coincide con el inicio de la cadena
  \item \texttt{\$} coincide con el final de la cadena
\end{itemize}

\begin{examplebox}[title=Usando Anclas]
\begin{lstlisting}[language=BCL]
SET text "hello world"

# Must start with "hello"
IF [REGEXP "^hello" $text] THEN
  PUTS "Starts with 'hello'"
END

# Must end with "world"
IF [REGEXP "world$" $text] THEN
  PUTS "Ends with 'world'"
END

# Must be EXACTLY "hello world" (nothing before or after)
IF [REGEXP "^hello world$" $text] THEN
  PUTS "Exact match"
END

# Won't match - has extra text
SET text2 "say hello world now"
IF [REGEXP "^hello world$" $text2] THEN
  PUTS "Exact match"
ELSE
  PUTS "Not an exact match - has extra text"
END
\end{lstlisting}

\textbf{Salida:}
\begin{verbatim}
Starts with 'hello'
Ends with 'world'
Exact match
Not an exact match - has extra text
\end{verbatim}
\end{examplebox}

\subsection{Clases de Caracteres Abreviadas}

BCL proporciona atajos para patrones comunes:

\begin{table}[h]
\centering
\caption{Clases de Caracteres Abreviadas}
\label{tab:regex_shorthand}
\begin{tabular}{lp{8cm}}
\toprule
\textbf{Abreviación} & \textbf{Equivalente a} \\
\midrule
\texttt{\textbackslash d} & \texttt{[0-9]} - cualquier dígito \\
\texttt{\textbackslash D} & \texttt{[\^{}0-9]} - cualquier no-dígito \\
\texttt{\textbackslash w} & \texttt{[a-zA-Z0-9\_]} - carácter de palabra \\
\texttt{\textbackslash W} & Carácter no-palabra \\
\texttt{\textbackslash s} & Espacio en blanco (espacio, tab, nueva línea) \\
\texttt{\textbackslash S} & No-espacio en blanco \\
\bottomrule
\end{tabular}
\end{table}

\begin{examplebox}[title=Ejemplos de Abreviaciones]
\begin{lstlisting}[language=BCL]
# Find digits with \d
SET text "Room 101 is on floor 5"
IF [REGEXP "\\d+" $text] THEN
  PUTS "Found numbers"
END

# Find words with \w
IF [REGEXP "\\w+" $text] THEN
  PUTS "Found word characters"
END

# Validate format: word space word
SET input "hello world"
IF [REGEXP "^\\w+\\s+\\w+$" $input] THEN
  PUTS "Valid format: two words separated by space"
END
\end{lstlisting}

\textbf{Salida:}
\begin{verbatim}
Found numbers
Found word characters
Valid format: two words separated by space
\end{verbatim}
\end{examplebox}

\begin{warningbox}
En BCL, necesitas escapar las barras invertidas en cadenas. Escribe \texttt{\textbackslash\textbackslash d} para representar \texttt{\textbackslash d} en el patrón.
\end{warningbox}

\section{El Comando REGEXP}

\cmd{REGEXP} verifica si un patrón coincide y puede extraer porciones coincidentes.

\textbf{Sintaxis:}
\begin{lstlisting}[language=BCL]
REGEXP pattern string                    # Returns 1 if match, 0 if not
REGEXP pattern string matchvar           # Store entire match
REGEXP pattern string matchvar subvar... # Store submatches
\end{lstlisting}

\subsection{Coincidencia Básica}

\begin{examplebox}[title=Coincidencia Simple de Patrones]
\begin{lstlisting}[language=BCL]
SET email "user@example.com"

# Check if it looks like an email
IF [REGEXP "@" $email] THEN
  PUTS "Contains @ symbol"
END

# Check for email pattern: word @ word . word
IF [REGEXP "\\w+@\\w+\\.\\w+" $email] THEN
  PUTS "Looks like a valid email"
END

# Validate phone number: exactly 10 digits
SET phone "5551234567"
IF [REGEXP "^\\d\{10\}$" $phone] THEN
  PUTS "Valid 10-digit phone number"
END
\end{lstlisting}

\textbf{Salida:}
\begin{verbatim}
Contains @ symbol
Looks like a valid email
Valid 10-digit phone number
\end{verbatim}
\end{examplebox}

\subsection{Capturar Coincidencias}

Usa paréntesis \texttt{(...)} para capturar partes de la coincidencia.

\begin{examplebox}[title=Extraer Información]
\begin{lstlisting}[language=BCL]
# Extract year from date
SET date "Today is 2025-10-22"
REGEXP "(\\d\{4\})-(\\d\{2\})-(\\d\{2\})" $date MATCH year month day
PUTS "Year: $year"
PUTS "Month: $month"
PUTS "Day: $day"

# Extract name and extension from filename
SET filename "document.pdf"
REGEXP "(.+)\\.(\\w+)$" $filename MATCH name ext
PUTS "Name: $name"
PUTS "Extension: $ext"

# Extract email parts
SET email "john.doe@example.com"
REGEXP "(.+)@(.+)" $email MATCH username domain
PUTS "Username: $username"
PUTS "Domain: $domain"
\end{lstlisting}

\textbf{Salida:}
\begin{verbatim}
Year: 2025
Month: 10
Day: 22
Name: document
Extension: pdf
Username: john.doe
Domain: example.com
\end{verbatim}
\end{examplebox}

\subsection{Opciones de REGEXP}

\begin{table}[h]
\centering
\caption{Opciones de REGEXP}
\label{tab:regexp_options}
\begin{tabular}{lp{8cm}}
\toprule
\textbf{Opción} & \textbf{Descripción} \\
\midrule
\texttt{NOCASE} & Coincidencia insensible a mayúsculas/minúsculas \\
\texttt{ALL} & Contar/coincidir todas las ocurrencias (devuelve contador) \\
\bottomrule
\end{tabular}
\end{table}

\begin{examplebox}[title=Coincidencia Insensible a Mayúsculas]
\begin{lstlisting}[language=BCL]
SET text "Hello World"

# Sensible a mayúsculas (no coincide)
IF [REGEXP "hello" $text] THEN
  PUTS "Found (sensitive)"
ELSE
  PUTS "Not found (sensitive)"
END

# Insensible a mayúsculas (coincide)
IF [REGEXP "hello" $text NOCASE] THEN
  PUTS "Found (insensitive)"
END

# Contar todas las ocurrencias
SET text2 "foo bar foo baz foo"
SET count [REGEXP "foo" $text2 ALL]
PUTS "Encontrado 'foo' $count veces"
\end{lstlisting}

\textbf{Salida:}
\begin{verbatim}
Not found (sensitive)
Found (insensitive)
Encontrado 'foo' 3 veces
\end{verbatim}
\end{examplebox}

\section{El Comando REGSUB}

\cmd{REGSUB} reemplaza texto que coincide con un patrón.

\textbf{Sintaxis:}
\begin{lstlisting}[language=BCL]
REGSUB pattern string replacement                # Reemplazar primera coincidencia
REGSUB pattern string replacement ALL            # Reemplazar todas las coincidencias
REGSUB pattern string replacement NOCASE         # Insensible a mayúsculas
REGSUB pattern string replacement ALL NOCASE     # Ambas opciones
\end{lstlisting}

\subsection{Reemplazo Básico}

\begin{examplebox}[title=Reemplazos Simples]
\begin{lstlisting}[language=BCL]
SET text "Hello, World!"

# Replace first occurrence
SET result [REGSUB "World" $text "BCL"]
PUTS $result  # Hello, BCL!

# Replace all occurrences
SET text2 "foo bar foo baz foo"
SET result2 [REGSUB "foo" $text2 "XXX" ALL]
PUTS $result2  # XXX bar XXX baz XXX

# Remove all digits
SET text3 "Room 101 is on floor 5"
SET clean [REGSUB "\\d+" $text3 "" ALL]
PUTS $clean  # Room  is on floor
\end{lstlisting}

\textbf{Salida:}
\begin{verbatim}
Hello, BCL!
XXX bar XXX baz XXX
Room  is on floor
\end{verbatim}
\end{examplebox}

\subsection{Usar Grupos Capturados en Reemplazo}

Puedes referenciar grupos capturados en el reemplazo usando \texttt{\&} o \texttt{\textbackslash 1}, \texttt{\textbackslash 2}, etc.

\begin{examplebox}[title=Reemplazos Avanzados]
\begin{lstlisting}[language=BCL]
# Swap first and last name
SET name "John Doe"
SET swapped [REGSUB "(\\w+) (\\w+)" $name "\\2, \\1"]
PUTS $swapped  # Doe, John

# Format phone number: 5551234567 -> (555) 123-4567
SET phone "5551234567"
SET formatted [REGSUB "(\\d\{3\})(\\d\{3\})(\\d\{4\})" $phone "(\\1) \\2-\\3"]
PUTS $formatted  # (555) 123-4567

# Add "http://" to URLs that don't have it
SET url "example.com"
IF [REGEXP "^http" $url] = 0 THEN
  SET url [REGSUB "^" $url "http://"]
END
PUTS $url  # http://example.com
\end{lstlisting}

\textbf{Salida:}
\begin{verbatim}
Doe, John
(555) 123-4567
http://example.com
\end{verbatim}
\end{examplebox}

\section{Ejemplos Prácticos}

\subsection{Validador de Email (Avanzado)}

\begin{examplebox}[title=Validación Completa de Email]
\begin{lstlisting}[language=BCL]
PROC validate_email WITH email DO
  # Basic pattern: user@domain.tld
  SET pattern "^\\w+(\\.\\w+)*@\\w+(\\.\\w+)+$"

  IF [REGEXP $pattern $email] THEN
    PUTS "$email is VALID"
    RETURN 1
  ELSE
    PUTS "$email is INVALID"
    RETURN 0
  END
END

# Test cases
validate_email "user@example.com"
validate_email "john.doe@company.co.uk"
validate_email "invalid@"
validate_email "@invalid.com"
validate_email "no-at-sign.com"
\end{lstlisting}

\textbf{Salida:}
\begin{verbatim}
user@example.com is VALID
john.doe@company.co.uk is VALID
invalid@ is INVALID
@invalid.com is INVALID
no-at-sign.com is INVALID
\end{verbatim}
\end{examplebox}

\subsection{Validador de Fortaleza de Contraseña}

\begin{examplebox}[title=Verificación de Contraseña Basada en Regex]
\begin{lstlisting}[language=BCL]
PROC check_password_strength WITH password DO
  SET score 0

  # Check length
  IF [STRING LENGTH $password] >= 8 THEN
    INCR score
  END

  # Check for lowercase
  IF [REGEXP "\[a-z\]" $password] THEN
    INCR score
  END

  # Check for uppercase
  IF [REGEXP "\[A-Z\]" $password] THEN
    INCR score
  END

  # Check for digits
  IF [REGEXP "\\d" $password] THEN
    INCR score
  END

  # Check for special characters
  IF [REGEXP "\[!@#$%^&*\]" $password] THEN
    INCR score
  END

  # Report strength
  IF $score < 3 THEN
    RETURN "Weak"
  ELSEIF $score < 5 THEN
    RETURN "Medium"
  ELSE
    RETURN "Strong"
  END
END

SET passwords [LIST "hello" "Hello123" "MyP@ss123" "Complex$Pass9"]
FOREACH pass IN $passwords DO
  SET strength [check_password_strength $pass]
  PUTS "$pass: $strength"
END
\end{lstlisting}

\textbf{Salida:}
\begin{verbatim}
hello: Weak
Hello123: Medium
MyP@ss123: Strong
Complex$Pass9: Strong
\end{verbatim}
\end{examplebox}

\subsection{Formatear Números de Teléfono}

\begin{examplebox}[title=Normalizar Números de Teléfono]
\begin{lstlisting}[language=BCL]
PROC format_phone WITH phone DO
  # Remove all non-digits
  SET clean [REGSUB "\\D" $phone "" ALL]

  # Check if we have 10 digits
  IF [STRING LENGTH $clean] != 10 THEN
    RETURN "Invalid phone number"
  END

  # Format as (XXX) XXX-XXXX
  SET formatted [REGSUB "(\\d\{3\})(\\d\{3\})(\\d\{4\})" $clean "(\\1) \\2-\\3"]
  RETURN $formatted
END

# Test with various formats
SET phones [LIST "5551234567" "555-123-4567" "(555) 123-4567" "555.123.4567"]
FOREACH phone IN $phones DO
  SET formatted [format_phone $phone]
  PUTS "$phone -> $formatted"
END
\end{lstlisting}

\textbf{Salida:}
\begin{verbatim}
5551234567 -> (555) 123-4567
555-123-4567 -> (555) 123-4567
(555) 123-4567 -> (555) 123-4567
555.123.4567 -> (555) 123-4567
\end{verbatim}
\end{examplebox}

\section{Patrones Comunes de Expresiones Regulares}

Aquí hay una referencia de patrones útiles para tareas comunes:

\begin{table}[h]
\centering
\caption{Patrones Regex Comunes}
\label{tab:common_patterns}
\begin{tabular}{lp{7cm}}
\toprule
\textbf{Patrón} & \textbf{Descripción} \\
\midrule
\texttt{\^{}\textbackslash d\{4\}-\textbackslash d\{2\}-\textbackslash d\{2\}\$} & Fecha (AAAA-MM-DD) \\
\texttt{\textbackslash w+@\textbackslash w+\textbackslash .\textbackslash w+} & Email simple \\
\texttt{\^{}\textbackslash d\{3\}-\textbackslash d\{3\}-\textbackslash d\{4\}\$} & Teléfono (XXX-XXX-XXXX) \\
\texttt{\^{}[01]?\textbackslash d\textbackslash d?\$} & Número 0-199 \\
\texttt{\textbackslash b\textbackslash w\{3\}\textbackslash b} & Palabra exactamente de 3 letras \\
\texttt{https?://.*} & URL HTTP o HTTPS \\
\texttt{\^{}\textbackslash s*\$} & Vacío o solo espacios en blanco \\
\texttt{\textbackslash d+\textbackslash .\textbackslash d\{2\}} & Decimal con 2 lugares \\
\bottomrule
\end{tabular}
\end{table}

\begin{tipbox}
Las expresiones regulares pueden volverse complejas. Comienza simple y construye gradualmente. Prueba tus patrones con varias entradas para asegurar que funcionen como se espera.
\end{tipbox}

\begin{warningbox}
Ten cuidado con patrones como \texttt{.*} (coincidir con cualquier cosa) ya que pueden coincidir con más de lo que esperas. Usa patrones específicos cuando sea posible.
\end{warningbox}
