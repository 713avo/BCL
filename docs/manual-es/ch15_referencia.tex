% ============================================================================
% CAPÍTULO 15: REFERENCIA
% ============================================================================

\chapter{Referencia Rápida}
\label{ch:reference}

Este capítulo proporciona una referencia alfabética rápida de todos los comandos BCL.

\begin{longtable}{p{3cm}p{10cm}}
\caption{Referencia de Comandos BCL} \label{tab:command_ref} \\
\toprule
\textbf{Comando} & \textbf{Descripción} \\
\midrule
\endfirsthead
\multicolumn{2}{c}{{\tablename\ \thetable{} -- continuación}} \\
\toprule
\textbf{Comando} & \textbf{Descripción} \\
\midrule
\endhead
\bottomrule
\endfoot

\cmd{AFTER} & Pausar ejecución por milisegundos \\
\cmd{APPEND} & Agregar texto a variable \\
\cmd{ARGV} & Obtener argumentos de línea de comandos \\
\cmd{BREAK} & Salir del bucle actual \\
\cmd{CLOCK} & Operaciones de tiempo y fecha \\
\cmd{CLOSE} & Cerrar manejador de archivo \\
\cmd{CONCAT} & Concatenar listas \\
\cmd{CONTINUE} & Saltar a la siguiente iteración del bucle \\
\cmd{ENV} & Acceder a variables de entorno \\
\cmd{EOF} & Verificar fin de archivo \\
\cmd{EVAL} & Evaluar cadena como código \\
\cmd{EXEC} & Ejecutar comando del sistema \\
\cmd{EXIT} & Terminar programa \\
\cmd{EXPR} & Evaluar expresión \\
\cmd{FILE} & Operaciones de archivo (EXISTS, SIZE, DELETE, RENAME) \\
\cmd{FOR} & Bucle for \\
\cmd{FOREACH} & Iterar sobre lista \\
\cmd{FORMAT} & Formatear cadena (estilo printf) \\
\cmd{GETS} & Leer línea de entrada \\
\cmd{GLOB} & Coincidencia de patrones de archivo \\
\cmd{GLOBAL} & Declarar variable global \\
\cmd{IF} & Ejecución condicional \\
\cmd{INCR} & Incrementar variable \\
\cmd{INFO} & Comandos de introspección \\
\cmd{JOIN} & Unir lista a cadena \\
\cmd{LAPPEND} & Agregar a lista \\
\cmd{LINDEX} & Obtener elemento de lista \\
\cmd{LINSERT} & Insertar en lista \\
\cmd{LIST} & Crear lista \\
\cmd{LLENGTH} & Obtener longitud de lista \\
\cmd{LRANGE} & Extraer sublista \\
\cmd{LREPLACE} & Reemplazar elementos de lista \\
\cmd{LSEARCH} & Buscar en lista \\
\cmd{LSORT} & Ordenar lista \\
\cmd{OPEN} & Abrir archivo \\
\cmd{PROC} & Definir procedimiento \\
\cmd{PUTS} & Imprimir con nueva línea \\
\cmd{PUTSN} & Imprimir sin nueva línea \\
\cmd{PWD} & Obtener directorio actual \\
\cmd{READ} & Leer de archivo \\
\cmd{REGEXP} & Coincidencia de expresión regular \\
\cmd{REGSUB} & Sustitución de expresión regular \\
\cmd{RETURN} & Retornar de procedimiento \\
\cmd{SCAN} & Analizar cadena formateada \\
\cmd{SEEK} & Establecer posición de archivo \\
\cmd{SET} & Asignar/leer variable \\
\cmd{SOURCE} & Ejecutar archivo de script \\
\cmd{SPLIT} & Dividir cadena a lista \\
\cmd{STRING} & Operaciones de cadena \\
\cmd{SWITCH} & Rama de múltiples vías \\
\cmd{TELL} & Obtener posición de archivo \\
\cmd{UNSET} & Eliminar variable \\
\cmd{WHILE} & Bucle while \\

\end{longtable}
