% ============================================================================
% CAPÍTULO 16: ARRAYS ASOCIATIVOS
% ============================================================================

\chapter{Arrays Asociativos}
\label{ch:arrays}

BCL soporta arrays asociativos (también conocidos como diccionarios o mapas hash en otros lenguajes), inspirados en la sintaxis de arrays de Tcl. Los arrays permiten almacenar valores indexados por claves arbitrarias, ya sean números o cadenas de texto.

\section{Sintaxis Básica}

\textbf{Asignación:}
\begin{lstlisting}[language=BCL]
SET nombreArray(indice) valor
\end{lstlisting}

\textbf{Acceso:}
\begin{lstlisting}[language=BCL]
$nombreArray(indice)
\end{lstlisting}

\begin{examplebox}[title=Array Simple]
\begin{lstlisting}[language=BCL]
SET frutas(1) "manzana"
SET frutas(2) "naranja"
SET frutas(3) "platano"

PUTS $frutas(1)    # manzana
PUTS $frutas(2)    # naranja
PUTS $frutas(3)    # platano
\end{lstlisting}

\textbf{Salida:}
\begin{verbatim}
manzana
naranja
plátano
\end{verbatim}
\end{examplebox}

\section{Índices Numéricos}

Los arrays pueden usar números como índices, similar a los arrays tradicionales:

\begin{examplebox}[title=Índices Numéricos con Bucle]
\begin{lstlisting}[language=BCL]
SET colores(1) "rojo"
SET colores(2) "verde"
SET colores(3) "azul"

SET i 1
WHILE $i <= 3 DO
  PUTS "colores($i) = $colores($i)"
  INCR i
END
\end{lstlisting}

\textbf{Salida:}
\begin{verbatim}
colores(1) = rojo
colores(2) = verde
colores(3) = azul
\end{verbatim}
\end{examplebox}

\section{Índices de Texto (Asociativos)}

El verdadero poder de los arrays de BCL viene de usar claves de texto:

\begin{examplebox}[title=Directorio Telefónico]
\begin{lstlisting}[language=BCL]
SET telefono(Juan) "555-1234"
SET telefono(Maria) "555-5678"
SET telefono(Pedro) "555-9012"

SET contacto "Maria"
PUTS "Telefono de $contacto: $telefono($contacto)"
\end{lstlisting}

\textbf{Salida:}
\begin{verbatim}
Teléfono de María: 555-5678
\end{verbatim}
\end{examplebox}

\section{Índices Variables}

Puedes usar variables como índices de array:

\begin{examplebox}[title=Variable como Índice]
\begin{lstlisting}[language=BCL]
SET datos(lunes) "10"
SET datos(martes) "15"
SET datos(miercoles) "12"

SET dia "martes"
PUTS "Datos del $dia: $datos($dia)"
\end{lstlisting}

\textbf{Salida:}
\begin{verbatim}
Datos del martes: 15
\end{verbatim}
\end{examplebox}

\section{Índices con Expresiones}

Los índices pueden ser expresiones, permitiendo acceso calculado al array:

\begin{examplebox}[title=Expresión como Índice]
\begin{lstlisting}[language=BCL]
SET tabla(1) "A"
SET tabla(2) "B"
SET tabla(3) "C"

SET i 1
SET j [EXPR $i + 1]
PUTS $tabla($j)    # B

SET k [EXPR $i * 3]
PUTS $tabla($k)    # C
\end{lstlisting}

\textbf{Salida:}
\begin{verbatim}
B
C
\end{verbatim}
\end{examplebox}

\section{Arrays Multidimensionales (Simulados)}

BCL simula arrays multidimensionales usando índices compuestos:

\begin{examplebox}[title=Matriz 2D]
\begin{lstlisting}[language=BCL]
SET matriz(1,1) "A"
SET matriz(1,2) "B"
SET matriz(2,1) "C"
SET matriz(2,2) "D"

PUTS "$matriz(1,1) $matriz(1,2)"
PUTS "$matriz(2,1) $matriz(2,2)"
\end{lstlisting}

\textbf{Salida:}
\begin{verbatim}
A B
C D
\end{verbatim}
\end{examplebox}

\section{Verificar Existencia de Elementos}

Usa \cmd{INFO EXISTS} para verificar si un elemento del array existe:

\begin{examplebox}[title=Verificar Existencia]
\begin{lstlisting}[language=BCL]
SET config(debug) "true"

IF [INFO EXISTS config(debug)] THEN
  PUTS "Debug es: $config(debug)"
ELSE
  PUTS "Debug no establecido"
END

IF [INFO EXISTS config(timeout)] THEN
  PUTS "Timeout es: $config(timeout)"
ELSE
  PUTS "Timeout no configurado"
END
\end{lstlisting}

\textbf{Salida:}
\begin{verbatim}
Debug es: true
Timeout no configurado
\end{verbatim}
\end{examplebox}

\section{Ejemplos Prácticos}

\subsection{Configuración de Aplicación}

\begin{examplebox}[title=Configuración]
\begin{lstlisting}[language=BCL]
SET config(host) "localhost"
SET config(port) "8080"
SET config(timeout) "30"
SET config(debug) "false"

PUTS "Servidor: $config(host):$config(port)"
PUTS "Timeout: $config(timeout)s"
PUTS "Modo debug: $config(debug)"
\end{lstlisting}

\textbf{Salida:}
\begin{verbatim}
Servidor: localhost:8080
Timeout: 30s
Modo debug: false
\end{verbatim}
\end{examplebox}

\subsection{Contador de Eventos}

\begin{examplebox}[title=Rastreo de Eventos]
\begin{lstlisting}[language=BCL]
SET contador(login) "0"
SET contador(logout) "0"
SET contador(error) "0"

# Simular eventos
SET contador(login) [EXPR $contador(login) + 1]
SET contador(login) [EXPR $contador(login) + 1]
SET contador(logout) [EXPR $contador(logout) + 1]
SET contador(error) [EXPR $contador(error) + 1]
SET contador(login) [EXPR $contador(login) + 1]

PUTS "Login: $contador(login)"
PUTS "Logout: $contador(logout)"
PUTS "Error: $contador(error)"
\end{lstlisting}

\textbf{Salida:}
\begin{verbatim}
Login: 3
Logout: 1
Error: 1
\end{verbatim}
\end{examplebox}

\subsection{Tabla de Multiplicar}

\begin{examplebox}[title=Tabla de Multiplicar]
\begin{lstlisting}[language=BCL]
SET num 5
SET i 1
WHILE $i <= 10 DO
  SET tabla($num,$i) [EXPR $num * $i]
  PUTS "$num x $i = $tabla($num,$i)"
  INCR i
END
\end{lstlisting}

\textbf{Salida:}
\begin{verbatim}
5 x 1 = 5
5 x 2 = 10
...
5 x 10 = 50
\end{verbatim}
\end{examplebox}

\section{Arrays vs Listas}

\begin{tipbox}
\textbf{Cuándo usar Arrays:}
\begin{itemize}
  \item Mapeos clave-valor (directorio telefónico, configuración)
  \item Índices arbitrarios (números no consecutivos o cadenas)
  \item Búsqueda rápida por clave
\end{itemize}

\textbf{Cuándo usar Listas:}
\begin{itemize}
  \item Colecciones ordenadas
  \item Procesamiento secuencial
  \item Operaciones como ordenar, unir, dividir
\end{itemize}
\end{tipbox}

\section{Arrays en Procedimientos}

Los arrays se pueden usar con la palabra clave \cmd{GLOBAL} en procedimientos:

\begin{examplebox}[title=Arrays Globales en Procedimientos]
\begin{lstlisting}[language=BCL]
PROC mostrar_persona DO
  GLOBAL persona
  PUTS "Nombre: $persona(nombre)"
  PUTS "Edad: $persona(edad)"
END

SET persona(nombre) "Alicia"
SET persona(edad) "30"
mostrar_persona
\end{lstlisting}

\textbf{Salida:}
\begin{verbatim}
Nombre: Alicia
Edad: 30
\end{verbatim}
\end{examplebox}

\begin{notebox}
Los elementos de array se almacenan como variables individuales con nombres como \texttt{nombrearray(indice)}. Se comportan como variables regulares y siguen las mismas reglas de alcance.
\end{notebox}

\section{El Comando ARRAY}

BCL proporciona el comando \cmd{ARRAY} para la manipulacion avanzada de arrays, inspirado en Tcl. Este comando ofrece operaciones poderosas para trabajar con arrays como estructuras completas.

\subsection{ARRAY EXISTS}

Verifica si un array existe (tiene al menos un elemento):

\begin{lstlisting}[language=BCL]
ARRAY EXISTS nombreArray
\end{lstlisting}

Devuelve \texttt{"1"} si el array existe, \texttt{"0"} en caso contrario.

\begin{examplebox}[title=Verificar Existencia de Array]
\begin{lstlisting}[language=BCL]
SET resultado [ARRAY EXISTS config]    # "0"
SET config(debug) "true"
SET resultado [ARRAY EXISTS config]    # "1"
\end{lstlisting}
\end{examplebox}

\subsection{ARRAY SIZE}

Obtiene el numero de elementos en un array:

\begin{lstlisting}[language=BCL]
ARRAY SIZE nombreArray
\end{lstlisting}

\begin{examplebox}[title=Tamano de Array]
\begin{lstlisting}[language=BCL]
SET colores(rojo) "#FF0000"
SET colores(verde) "#00FF00"
SET colores(azul) "#0000FF"

SET tam [ARRAY SIZE colores]       # "3"
PUTS "El array tiene $tam elementos"
\end{lstlisting}

\textbf{Salida:}
\begin{verbatim}
El array tiene 3 elementos
\end{verbatim}
\end{examplebox}

\subsection{ARRAY NAMES}

Obtiene una lista de indices del array, opcionalmente filtrados por un patron:

\begin{lstlisting}[language=BCL]
ARRAY NAMES nombreArray ?patron?
\end{lstlisting}

Los patrones soportan comodines glob: \texttt{*} (cualquier caracter), \texttt{?} (un caracter), \texttt{[abc]} (conjunto de caracteres).

\begin{examplebox}[title=Listar Indices de Array]
\begin{lstlisting}[language=BCL]
SET datos(nombre) "Juan"
SET datos(edad) "30"
SET datos(nota1) "8"
SET datos(nota2) "9"

SET todos [ARRAY NAMES datos]            # Todos los indices
SET notas [ARRAY NAMES datos "nota*"]    # Solo notas
SET con_a [ARRAY NAMES datos "*a*"]      # Que contengan 'a'

PUTS "Todos: $todos"
PUTS "Notas: $notas"
PUTS "Con 'a': $con_a"
\end{lstlisting}

\textbf{Salida:}
\begin{verbatim}
Todos: nombre edad nota1 nota2
Notas: nota1 nota2
Con 'a': nombre edad nota1 nota2
\end{verbatim}
\end{examplebox}

\subsection{ARRAY GET}

Obtiene el contenido del array como una lista de pares indice-valor:

\begin{lstlisting}[language=BCL]
ARRAY GET nombreArray ?patron?
\end{lstlisting}

Devuelve indices y valores alternados, opcionalmente filtrados por patron.

\begin{examplebox}[title=Obtener Contenido de Array]
\begin{lstlisting}[language=BCL]
SET colores(rojo) "255,0,0"
SET colores(verde) "0,255,0"
SET colores(azul) "0,0,255"

SET todos [ARRAY GET colores]
PUTS "Todos los colores: $todos"
\end{lstlisting}

\textbf{Salida:}
\begin{verbatim}
Todos los colores: rojo 255,0,0 verde 0,255,0 azul 0,0,255
\end{verbatim}
\end{examplebox}

\subsection{ARRAY SET}

Puebla un array desde una lista de pares indice-valor:

\begin{lstlisting}[language=BCL]
ARRAY SET nombreArray lista
\end{lstlisting}

La lista debe tener un numero par de elementos.

\begin{examplebox}[title=Establecer Array desde Lista]
\begin{lstlisting}[language=BCL]
# Crear array desde lista
ARRAY SET config "host localhost port 8080 debug true"

PUTS "Host: $config(host)"
PUTS "Puerto: $config(port)"
PUTS "Debug: $config(debug)"
\end{lstlisting}

\textbf{Salida:}
\begin{verbatim}
Host: localhost
Puerto: 8080
Debug: true
\end{verbatim}
\end{examplebox}

\textbf{Copiar Arrays:}

\begin{examplebox}[title=Copiar un Array]
\begin{lstlisting}[language=BCL]
SET original(a) "1"
SET original(b) "2"
SET original(c) "3"

# Copiar array
SET datos [ARRAY GET original]
ARRAY SET copia $datos

PUTS "Tamano copia: [ARRAY SIZE copia]"
PUTS "copia(a) = $copia(a)"
\end{lstlisting}

\textbf{Salida:}
\begin{verbatim}
Tamano copia: 3
copia(a) = 1
\end{verbatim}
\end{examplebox}

\subsection{ARRAY UNSET}

Elimina elementos del array que coincidan con un patron:

\begin{lstlisting}[language=BCL]
ARRAY UNSET nombreArray ?patron?
\end{lstlisting}

Si no se especifica patron, elimina el array completo.

\begin{examplebox}[title=Eliminacion Selectiva]
\begin{lstlisting}[language=BCL]
SET cache(temp_1) "data1"
SET cache(temp_2) "data2"
SET cache(perm_1) "data3"

PUTS "Antes: [ARRAY SIZE cache]"    # "3"
ARRAY UNSET cache "temp_*"
PUTS "Despues: [ARRAY SIZE cache]"  # "1"
\end{lstlisting}

\textbf{Salida:}
\begin{verbatim}
Antes: 3
Despues: 1
\end{verbatim}
\end{examplebox}

\begin{examplebox}[title=Eliminar Array Completo]
\begin{lstlisting}[language=BCL]
SET miarray(1) "valor1"
SET miarray(2) "valor2"

ARRAY UNSET miarray

SET existe [ARRAY EXISTS miarray]    # "0"
PUTS "Array existe: $existe"
\end{lstlisting}

\textbf{Salida:}
\begin{verbatim}
Array existe: 0
\end{verbatim}
\end{examplebox}

\subsection{Ejemplo Practico: Gestor de Configuracion}

\begin{examplebox}[title=Gestor de Configuracion]
\begin{lstlisting}[language=BCL]
# Cargar configuracion
ARRAY SET config "
    db_host localhost
    db_port 3306
    db_name myapp
    cache_enabled true
    cache_ttl 300
"

# Mostrar configuracion de base de datos
PUTS "Configuracion Base de Datos:"
SET db_keys [ARRAY NAMES config "db_*"]
SET i 0
WHILE $i < [LLENGTH $db_keys] DO
    SET key [LINDEX $db_keys $i]
    PUTS "  $key = $config($key)"
    INCR i
END

# Mostrar configuracion de cache
PUTS ""
PUTS "Configuracion Cache:"
SET cache_keys [ARRAY NAMES config "cache_*"]
SET i 0
WHILE $i < [LLENGTH $cache_keys] DO
    SET key [LINDEX $cache_keys $i]
    PUTS "  $key = $config($key)"
    INCR i
END
\end{lstlisting}

\textbf{Salida:}
\begin{verbatim}
Configuracion Base de Datos:
  db_host = localhost
  db_port = 3306
  db_name = myapp

Configuracion Cache:
  cache_enabled = true
  cache_ttl = 300
\end{verbatim}
\end{examplebox}
