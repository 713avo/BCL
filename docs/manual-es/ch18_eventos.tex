\chapter{Sistema de Eventos}
\label{ch:eventos}

BCL v2.0 introduce un sistema completo de programación orientada a eventos que permite operaciones de E/S asíncronas y callbacks basados en temporizadores. El sistema de eventos es similar al mecanismo \texttt{fileevent} de Tcl e se integra perfectamente con el sistema de extensiones dinámicas.

\section{Descripción General}

El sistema de eventos proporciona una manera de registrar callbacks que se ejecutan cuando ocurren eventos específicos:

\begin{itemize}
    \item \textbf{Eventos de E/S}: Descriptores de archivo que se vuelven legibles, escribibles o encuentran excepciones
    \item \textbf{Eventos de Temporizador}: Callbacks disparados después de un retraso especificado (milisegundos)
    \item \textbf{Eventos Personalizados}: Eventos definidos por extensiones (GPIO, señales, etc.)
\end{itemize}

\subsection{Arquitectura}

El sistema de eventos de BCL usa una arquitectura híbrida:

\begin{itemize}
    \item \textbf{Núcleo}: POSIX \texttt{select()} para multiplexación de E/S portable
    \item \textbf{Temporizadores}: Temporizadores de alta precisión en milisegundos
    \item \textbf{Extensiones}: Backends opcionales (epoll, kqueue) para alto rendimiento
\end{itemize}

\section{Comando EVENT}

El comando \texttt{EVENT} tiene seis subcomandos para gestionar el bucle de eventos:

\subsection{EVENT CREATE - Registrar Evento de E/S}

\begin{lstlisting}[style=bcl]
EVENT CREATE fd tipo_evento proc_callback
\end{lstlisting}

Registra un callback para un evento de descriptor de archivo.

\textbf{Parámetros:}
\begin{itemize}
    \item \texttt{fd} - Número de descriptor de archivo
    \item \texttt{tipo\_evento} - READABLE, WRITABLE o EXCEPTION
    \item \texttt{proc\_callback} - Nombre del procedimiento (recibe fd como parámetro)
\end{itemize}

\textbf{Ejemplo:}
\begin{lstlisting}[style=bcl]
PROC AL_RECIBIR_DATOS WITH fd DO
    SET datos [READ $fd 1024]
    PUTS "Recibido: $datos"
END

EVENT CREATE $socket READABLE AL_RECIBIR_DATOS
\end{lstlisting}

\subsection{EVENT DELETE - Eliminar Evento}

\begin{lstlisting}[style=bcl]
EVENT DELETE fd [tipo_evento]
\end{lstlisting}

Elimina un manejador de eventos registrado.

\subsection{EVENT TIMER - Registrar Temporizador}

\begin{lstlisting}[style=bcl]
EVENT TIMER milisegundos proc_callback
\end{lstlisting}

Registra un callback de temporizador de disparo único.

\textbf{Ejemplo:}
\begin{lstlisting}[style=bcl]
PROC AL_TIMEOUT DO
    PUTS "Han pasado 5 segundos"
END

EVENT TIMER 5000 AL_TIMEOUT
\end{lstlisting}

\subsection{EVENT PROCESS - Procesar Una Vez}

\begin{lstlisting}[style=bcl]
EVENT PROCESS [timeout_ms]
\end{lstlisting}

Procesa eventos pendientes una vez, con timeout opcional.

\textbf{Retorna:} 1 si se procesó un evento, 0 en timeout

\subsection{EVENT LOOP - Ejecutar Bucle de Eventos}

\begin{lstlisting}[style=bcl]
EVENT LOOP
\end{lstlisting}

Ejecuta el bucle de eventos indefinidamente hasta que se llame \texttt{EVENT STOP}.

\subsection{EVENT STOP - Detener Bucle de Eventos}

\begin{lstlisting}[style=bcl]
EVENT STOP
\end{lstlisting}

Detiene el bucle de eventos (hace que \texttt{EVENT LOOP} retorne).

\subsection{EVENT INFO - Listar Eventos}

\begin{lstlisting}[style=bcl]
SET info [EVENT INFO]
\end{lstlisting}

Retorna una lista de todos los eventos registrados en formato legible.

\section{Ejemplos Completos}

\subsection{Servidor Echo TCP}

\begin{lstlisting}[style=bcl]
LOAD "extensions/socket.so"

GLOBAL clientes
SET clientes [LIST]

PROC AL_ACEPTAR WITH servidor_fd DO
    GLOBAL clientes
    SET cliente [SOCKET ACCEPT $servidor_fd]
    SET clientes [LAPPEND $clientes $cliente]
    PUTS "Cliente $cliente conectado"
    
    EVENT CREATE $cliente READABLE AL_RECIBIR_DATOS
END

PROC AL_RECIBIR_DATOS WITH cliente_fd DO
    GLOBAL clientes
    SET datos [SOCKET RECV $cliente_fd 1024]
    
    IF [EXPR [STRING LENGTH $datos] == 0] THEN
        PUTS "Cliente $cliente_fd desconectado"
        SOCKET CLOSE $cliente_fd
        EVENT DELETE $cliente_fd
        SET clientes [LREMOVE $clientes $cliente_fd]
    ELSE
        SOCKET SEND $cliente_fd $datos
    END
END

SET servidor [SOCKET SERVER 9999]
EVENT CREATE $servidor READABLE AL_ACEPTAR

PUTS "Servidor echo escuchando en puerto 9999"
EVENT LOOP
\end{lstlisting}

\subsection{Temporizador Repetitivo}

\begin{lstlisting}[style=bcl]
GLOBAL contador
SET contador 0

PROC AL_TICK DO
    GLOBAL contador
    INCR contador
    
    PUTS "Tick $contador"
    
    IF [EXPR $contador < 10] THEN
        EVENT TIMER 1000 AL_TICK
    ELSE
        PUTS "Terminado!"
        EVENT STOP
    END
END

EVENT TIMER 1000 AL_TICK
EVENT LOOP
\end{lstlisting}

\section{Integración con Extensiones}

El sistema de eventos está diseñado para trabajar con extensiones dinámicas. Las extensiones pueden:

\begin{enumerate}
    \item Retornar descriptores de archivo que pueden ser monitoreados con \texttt{EVENT CREATE}
    \item Registrar backends de eventos personalizados (epoll, kqueue, etc.)
    \item Proporcionar GPIO, señales u otras fuentes de eventos
\end{enumerate}

\section{Mejores Prácticas}

\subsection{Parámetros de Callback}

Los callbacks de eventos de E/S reciben el descriptor de archivo como parámetro:

\begin{lstlisting}[style=bcl]
# Correcto - usa parámetro
PROC AL_LEER WITH fd DO
    SET datos [READ $fd 1024]
END

# Incorrecto - depende de variable global
PROC AL_LEER DO
    GLOBAL socket
    SET datos [READ $socket 1024]
END
\end{lstlisting}

\subsection{Manejo de Errores}

Siempre verificar errores y limpiar:

\begin{lstlisting}[style=bcl]
PROC AL_RECIBIR WITH fd DO
    SET datos [SOCKET RECV $fd 1024]
    
    IF [EXPR [STRING LENGTH $datos] == 0] THEN
        SOCKET CLOSE $fd
        EVENT DELETE $fd
    ELSE
        # Procesar datos
    END
END
\end{lstlisting}

\section{Véase También}

\begin{itemize}
    \item Capítulo~\ref{ch:sistema}: Comandos de Integración del Sistema
    \item \texttt{docs/EVENT\_SYSTEM.md}: Documentación completa del sistema de eventos
    \item \texttt{docs/extensions/main.pdf}: Guía de desarrollo de extensiones
    \item \texttt{extensions/README.md}: Documentación de la extensión SOCKET
\end{itemize}
