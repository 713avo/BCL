% ============================================================================
% PÁGINAS PRELIMINARES
% ============================================================================

\frontmatter

% Página de título
\begin{titlepage}
  \centering
  \vspace*{2cm}

  {\Huge\bfseries Manual de Usuario de BCL\par}
  \vspace{0.5cm}
  {\Large Basic Command Language\par}
  \vspace{2cm}

  {\Large Versión 1.5.1\par}
  \vspace{1cm}

  {\large Octubre 2025\par}
  \vspace{2cm}

  {\Large\itshape Rafa\par}
  \vfill

  {\large Un lenguaje de scripting inspirado en Tcl\\
  con sintaxis estilo BASIC\par}

  \vspace{1cm}

  {\large Diseñado para sistemas embebidos y entornos PC\par}

  \vspace{2cm}
\end{titlepage}

% Página de copyright
\thispagestyle{empty}
\null\vfill
\noindent
Copyright \copyright{} 2025 Rafa\\

\noindent
Este manual describe BCL (Basic Command Language) versión 1.5.1.\\

\noindent
BCL está inspirado en Tcl 8.x pero presenta una sintaxis simplificada estilo BASIC, diseñada para ser amigable para principiantes y adecuada tanto para sistemas embebidos como para entornos PC.\\

\noindent
Se concede permiso para copiar, distribuir y/o modificar este documento con fines educativos y no comerciales.\\

\clearpage

% Tabla de contenidos
\tableofcontents

% Lista de tablas
\listoftables

% Lista de listados
\lstlistoflistings
